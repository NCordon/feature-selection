\documentclass[a4paper,11pt]{article}
% Símbolo del euro
\usepackage[gen]{eurosym}
% Codificación
\usepackage[utf8]{inputenc}
% Idioma
\usepackage[spanish]{babel} % English language/hyphenation
\selectlanguage{spanish}
% Hay que pelearse con babel-spanish para el alineamiento del punto decimal
\decimalpoint
\usepackage{dcolumn}
\newcolumntype{d}[1]{D{.}{\esperiod}{#1}}
\makeatletter
\addto\shorthandsspanish{\let\esperiod\es@period@code}
\makeatother

\usepackage[usenames,dvipsnames]{color} % Coloring code

\usepackage{csvsimple}
\usepackage{adjustbox}
\newsavebox\ltmcbox


% Para matrices
\usepackage{amsmath}

% Símbolos matemáticos
\usepackage{amssymb}
\let\oldemptyset\emptyset
\let\emptyset\varnothing

% Hipervínculos
\usepackage{url}

\usepackage[section]{placeins} % Para gráficas en su sección.
\usepackage{mathpazo} % Use the Palatino font
\usepackage[T1]{fontenc} % Required for accented characters
\newenvironment{allintypewriter}{\ttfamily}{\par}
\setlength{\parindent}{0pt}
\parskip=8pt
\linespread{1.05} % Change line spacing here, Palatino benefits from a slight increase by default


% Imágenes
\usepackage{graphicx}
\usepackage{float}
\usepackage{caption}
\usepackage{wrapfig} % Allows in-line images



% Márgenes
\usepackage{geometry}
 \geometry{
 a4paper,
 total={210mm,297mm},
 left=30mm,
 right=30mm,
 top=25mm,
 bottom=25mm,
 }


% Referencias
\usepackage{fncylab}
\labelformat{figure}{\textit{\figurename\space #1}}

\usepackage{hyperref}
\hypersetup{
  colorlinks   = true, %Colours links instead of ugly boxes
  urlcolor     = blue, %Colour for external hyperlinks
  linkcolor    = blue, %Colour of internal links
  citecolor   = red %Colour of citations
}


\makeatletter
\renewcommand{\@listI}{\itemsep=0pt} % Reduce the space between items in the itemize and enumerate environments and the bibliography
\newcommand{\imagent}[4]{
  \begin{wrapfigure}{#4}{0.7\textwidth}
    \begin{center}
    \includegraphics[width=0.7\textwidth]{#1}
    \end{center}
    \caption{#3}
    \label{#4}
  \end{wrapfigure}
}


\newcommand{\imagen}[4]{
  \begin{minipage}{\linewidth}
    \centering
    \includegraphics[width=#4\textwidth]{#1}
    \captionof{figure}{#2}
    \label{#3}
  \end{minipage} 
}

\newcommand{\imgn}[3]{
  \begin{minipage}{\linewidth}
    \centering
    \includegraphics[width=#3\textwidth]{#1}
    \captionof{figure}{#2}
  \end{minipage} 
}

% Ejemplo de parámetro: ILS.r
\newcommand{\hrefr}[1]{
\href{../bin/#1}{#1}
}

%Customize enumerate tag
\usepackage{enumitem}
%Sections don't get numbered
%\setcounter{secnumdepth}{0}
\newcommand{\blue}[1]{\textcolor{blue}{#1}}
\begin{document}
%%%%%%%%%%%%%%%%%%%%%%%%%%%%%%%%%%%%%%%%%
% University Assignment Title Page 
% LaTeX Template
% Version 1.0 (27/12/12)
%
% This template has been downloaded from:
% http://www.LaTeXTemplates.com
%
% Original author:
% WikiBooks (http://en.wikibooks.org/wiki/LaTeX/Title_Creation)
% Modified by: NCordon (https://github.com/NCordon)
%
% License:
% CC BY-NC-SA 3.0 (http://creativecommons.org/licenses/by-nc-sa/3.0/)
% 
% Instructions for using this template:
% This title page is capable of being compiled as is. This is not useful for 
% including it in another document. To do this, you have two options: 
%
% 1) Copy/paste everything between \begin{document} and \end{document} 
% starting at \begin{titlepage} and paste this into another LaTeX file where you 
% want your title page.
% OR
% 2) Remove everything outside the \begin{titlepage} and \end{titlepage} and 
% move this file to the same directory as the LaTeX file you wish to add it to. 
% Then add \input{./title_page_1.tex} to your LaTeX file where you want your
% title page.
%
%%%%%%%%%%%%%%%%%%%%%%%%%%%%%%%%%%%%%%%%%
\begin{titlepage}

\newcommand{\HRule}{\rule{\linewidth}{0.5mm}} % Defines a new command for the horizontal lines, change thickness here

\center % Center everything on the page
 
%----------------------------------------------------------------------------------------
%	HEADING SECTIONS
%----------------------------------------------------------------------------------------
\textsc{\LARGE Universidad de Granada}\\[1.5cm]
\textsc{\Large Metaheaurísticas}\\[0.5cm] 

%----------------------------------------------------------------------------------------
%	TITLE SECTION
%----------------------------------------------------------------------------------------
\bigskip
\HRule \\[0.4cm]
{ \huge \bfseries Práctica II}\\[0.4cm] % Title of your document
{ \huge \bfseries Selección de características}\\
\HRule \\[1.5cm]
 
%----------------------------------------------------------------------------------------
%	AUTHOR SECTION
%----------------------------------------------------------------------------------------

\begin{minipage}{\textwidth}
\begin{center} \large
\emph{Búsqueda Multiarranque Básica}\\
\emph{GRASP}\\
\emph{Búsqueda Local Reiterada(ILS)}\\
\end{center}
\end{minipage}

%----------------------------------------------------------------------------------------
%	LOGO SECTION
%----------------------------------------------------------------------------------------

\begin{center}
\includegraphics[width=8cm]{../data/ugr.jpg}
\end{center}
%----------------------------------------------------------------------------------------

\begin{minipage}{\textwidth}
\begin{center} \large
Ignacio Cordón Castillo, 25352973G\\
\url{nachocordon@correo.ugr.es}\\
\ \\
$4^{\circ}$ Doble Grado Matemáticas Informática\\
Grupo Prácticas Viernes
\end{center}
\end{minipage}


\vspace{\fill}% Fill the rest of the page with whitespace
\large\today
\end{titlepage}  

\newpage
\tableofcontents
\newpage
% Examples of inclussion of images
%\imagent{ugr.jpg}{Logo de prueba}{ugr}
%\imagen{ugr.jpg}{Logo de prueba}{ugr2}{size relative to the \textwidth}

\section{Descripción del problema}
Dado un \textit{dataset} de instancias ya clasificadas, con una serie de atributos, se pretende comparar las distintas 
metaheaurísticas disponibles para comprobar cuál produce el conjunto de atributos que sirven para obtener una mayor 
tasa de clasificación (número de instancias bien clasificadas sobre el total) usando un clasificador de instancias.
En otras palabras, si tenemos $p$ atributos o características, el número total de subconjuntos de características que podemos
escoger es $2^p$. Cada uno de esos conjuntos o selecciones de características tendrá una tasa de clasificación asociada para
el clasificador que hemos escogido. Como este problema cuando $p$ es muy grande es inabordable por fuerza bruta, se trata de aplicar metaheurísticas para encontrar la mejor selección de características que
podamos de entre esas $2^p$ posibilidades.

Se considera la tasa de clasificación en el problema usando un clasificador \textit{3-knn} leaving one-out. Para cada 
instancia de un conjunto, toma las 3 más cercanas usando la distancia euclídea entre sus atributos, y etiqueta la nueva
instancia en función de la etiqueta mayoritaria de entre esas tres. Se efectúan 5 particiones al 50\% estratificadas
por clase en \textit{train - test}, de modo que la metaheurística proporcionará un conjunto de atributos para el conjunto de
entrenamiento, y se calculará la tasa de clasificación que produce sobre el conjunto de prueba para el \textit{3-knn} dicho
conjunto de atributos. El proceso se repite intercambiando los conjuntos de entrenamiento y de test. La calidad de la 
metaheaurística/algoritmo se calculará como la media de todas las tasas de clasificación obtenidas sobre el conjunto test
(10 en total). Otras medidas que se tendrán en cuenta a la hora de evaluar la bondad de un algoritmo empleado serán:
\begin{itemize}
 \item Tasa de reducción: Porcentaje de características que ha eliminado la máscara sobre el total que podía 
 seleccionar. Mejor cuanto más alta. En nuestros resultados lo hemos expresado como un tanto por uno.
 \item Tiempo de ejecución: Tiempo en segundos que tarda el algoritmo en devolver un conjunto de características.
 Se busca el menor tiempo de ejecución posible.
\end{itemize}


La tasa de clasificación del clasificador se mide (en tanto por uno) como: $$tasa\: clasificacion = \frac{instancias\: 
bien\: clasificadas}{total\: instancias}$$

La representación usada es la binaria (1 o 0 por característica), donde tenemos un vector de tamaño $n$, con $n$ el número 
de atributos (exceptuando la clase), y la metaheaurística/algoritmo ha seleccionado el atributo si y solo si lo ha marcado 
a 1. A una representación de esta forma, la denominamos máscara: $$ mask =\begin{matrix} (0 & 1\ldots 1 & 1 & 0\ldots) 
\end{matrix}$$

\newpage
\section{Descripción elementos comunes de los algoritmos}
Los algoritmos considerados tienen las siguientes componentes:

\begin{itemize} 
\item \textbf{Esquema de representación:} se usarán máscaras, sucesiones de unos y ceros de tanta longitud como atributos haya,
exceptuando la clase, donde un 1 en la posición $i-$ésima indica que esa característica se ha escogido, y un $0$ que no.

\item \textbf{Función objetivo}: la función a maximizar será la tasa de clasificación explicada arriba usando el clasificador
3-knn con la selección de características indicada por la máscara.\\

\small\texttt{% Generator: GNU source-highlight, by Lorenzo Bettini, http://www.gnu.org/software/src-highlite
\noindent
\mbox{}\textbf{\textcolor{Blue}{def}}\ tasa$\_$clasificacion\ \textcolor{BrickRed}{(}train\textcolor{BrickRed}{,}\ mascara\textcolor{BrickRed}{)}\textcolor{Red}{\{} \\
\mbox{}\ \ train\ \textcolor{BrickRed}{=}\ apply\textcolor{BrickRed}{(}mascara\textcolor{BrickRed}{,}\ train\textcolor{BrickRed}{)} \\
\mbox{}\ \ clases\ \textcolor{BrickRed}{=}\ train\textcolor{BrickRed}{[}\textbf{\textcolor{Blue}{class}}\textcolor{BrickRed}{]} \\
\mbox{}\ \ train\ \textcolor{BrickRed}{=}\ train\textcolor{BrickRed}{[-}\textbf{\textcolor{Blue}{class}}\textcolor{BrickRed}{]} \\
\mbox{} \\
\mbox{}\ \ fit\ \textcolor{BrickRed}{=}\ leave\textcolor{BrickRed}{.}one\textcolor{BrickRed}{.}out\textcolor{BrickRed}{.}3knn\textcolor{BrickRed}{(}train\textcolor{BrickRed}{,}\ clases\textcolor{BrickRed}{)} \\
\mbox{} \\
\mbox{}\ \ \textbf{\textcolor{Blue}{return}}\ \textcolor{BrickRed}{(}\textcolor{Purple}{100}\ \textcolor{BrickRed}{*}\ length\textcolor{BrickRed}{(}which\textcolor{BrickRed}{(}clases\ \textcolor{BrickRed}{==}\ fit\textcolor{BrickRed}{))}\ \textcolor{BrickRed}{/}\ length\textcolor{BrickRed}{(}clases\textcolor{BrickRed}{))} \\
\mbox{}\textcolor{Red}{\}} \\
\mbox{}
}

\item \textbf{Generación de solución inicial: }
  \begin{itemize}
   \item BMB: se generarán iterativamente 25 soluciones aleatorias sobre las que se aplicará búsqueda local
   para refinarlas.\\
   
    \small\texttt{% Generator: GNU source-highlight, by Lorenzo Bettini, http://www.gnu.org/software/src-highlite
\noindent
\mbox{}\textbf{\textcolor{Blue}{def}}\ random$\_$mask\textcolor{BrickRed}{(}n\textcolor{BrickRed}{)}\textcolor{Red}{\{} \\
\mbox{}\ \ \ \ mascaras\ \textcolor{BrickRed}{=}\ \textcolor{BrickRed}{[}\ \textcolor{BrickRed}{[}\ random\textcolor{BrickRed}{(}\textcolor{Red}{\{}\textcolor{Purple}{0}\textcolor{BrickRed}{,}\textcolor{Purple}{1}\textcolor{Red}{\}}\textcolor{BrickRed}{)}\ from\ \textcolor{Purple}{0}\ to\ n\textcolor{BrickRed}{]}\ from\ \textcolor{Purple}{0}\ to\ max$\_$arranques\ \textcolor{BrickRed}{]} \\
\mbox{}\textcolor{Red}{\}} \\
\mbox{}
}
   
   \item GRASP: la solución inicial se generará mediante un \textit{greedy} aleatorizado, y se refinará mediante
   la búsqueda local, quedándonos con la mejor de entre 25 generadas.\\
   
   \small\texttt{% Generator: GNU source-highlight, by Lorenzo Bettini, http://www.gnu.org/software/src-highlite
\noindent
\mbox{}\textbf{\textcolor{Blue}{def}}\ random$\_$greedy\textcolor{BrickRed}{(}n\textcolor{BrickRed}{)}\textcolor{Red}{\{} \\
\mbox{}\ \ \ \ non$\_$selected\ \textcolor{BrickRed}{=}\ \textcolor{Red}{\{}\textcolor{Purple}{1}\textcolor{BrickRed}{,...}n\textcolor{Red}{\}} \\
\mbox{}\ \ \ \ mejor$\_$mascara\ \textcolor{BrickRed}{=}\ \textcolor{Red}{\{}\textcolor{Purple}{0}\textcolor{BrickRed}{,}\textcolor{Purple}{0}\textcolor{BrickRed}{...}\textcolor{Purple}{0}\textcolor{Red}{\}} \\
\mbox{}\ \ \ \ mejora\ \textcolor{BrickRed}{=}\ \textbf{\textcolor{Blue}{true}} \\
\mbox{} \\
\mbox{}\ \ \ \ \textbf{\textcolor{Blue}{do}}\textcolor{Red}{\{} \\
\mbox{}\ \ \ \ \ \ \ \ mascaras\ \textcolor{BrickRed}{=}\ \textcolor{Red}{\{\}} \\
\mbox{}\ \ \ \ \ \ \ \ tasas\ \textcolor{BrickRed}{=}\ \textcolor{Red}{\{\}} \\
\mbox{} \\
\mbox{}\ \ \ \ \ \ \ \ foreach\ j\ \textbf{\textcolor{Blue}{in}}\ non$\_$selected\textcolor{Red}{\{} \\
\mbox{}\ \ \ \ \ \ \ \ \ \ \ \ m\ \textcolor{BrickRed}{=}\ flip\textcolor{BrickRed}{(}mascara\textcolor{BrickRed}{,}j\textcolor{BrickRed}{)} \\
\mbox{}\ \ \ \ \ \ \ \ \ \ \ \ mascaras\textcolor{BrickRed}{.}add\textcolor{BrickRed}{(}m\textcolor{BrickRed}{)} \\
\mbox{}\ \ \ \ \ \ \ \ \ \ \ \ tasas\textcolor{BrickRed}{.}add\textcolor{BrickRed}{(}tasa\textcolor{BrickRed}{(}m\textcolor{BrickRed}{))} \\
\mbox{}\ \ \ \ \ \ \ \ \textcolor{Red}{\}} \\
\mbox{} \\
\mbox{}\ \ \ \ \ \ \ \ umbral\ \textcolor{BrickRed}{=}\ alpha\ \textcolor{BrickRed}{*}\ \textcolor{BrickRed}{(}max\textcolor{BrickRed}{(}tasas\textcolor{BrickRed}{)}\ \textcolor{BrickRed}{-}\ min\textcolor{BrickRed}{(}tasas\textcolor{BrickRed}{))} \\
\mbox{}\ \ \ \ \ \ \ \ mascaras\textcolor{BrickRed}{.}delete\ \textcolor{BrickRed}{(}which\textcolor{BrickRed}{(}max\textcolor{BrickRed}{(}tasas\textcolor{BrickRed}{)}\ \textcolor{BrickRed}{-}\ tasas\ \textcolor{BrickRed}{\textgreater{}}\ umbral\textcolor{BrickRed}{))} \\
\mbox{}\ \ \ \ \ \ \ \ m\ \textcolor{BrickRed}{=}\ mascaras\textcolor{BrickRed}{[}\ random\textcolor{Red}{\{}\textcolor{Purple}{1}\textcolor{BrickRed}{..}n\textcolor{Red}{\}}\ \textcolor{BrickRed}{]} \\
\mbox{} \\
\mbox{}\ \ \ \ \ \ \ \ \textbf{\textcolor{Blue}{if}}\ \textcolor{BrickRed}{(}tasa\textcolor{BrickRed}{(}m\textcolor{BrickRed}{)}\ \textcolor{BrickRed}{\textgreater{}}\ tasa\textcolor{BrickRed}{(}mejor$\_$mascara\textcolor{BrickRed}{))}\textcolor{Red}{\{} \\
\mbox{}\ \ \ \ \ \ \ \ \ \ \ \ mejor$\_$mascara\ \textcolor{BrickRed}{=}\ m \\
\mbox{}\ \ \ \ \ \ \ \ \ \ \ \ non$\_$selected\textcolor{BrickRed}{.}delete\textcolor{BrickRed}{(}j\ \textcolor{BrickRed}{:}\ flip\textcolor{BrickRed}{(}mascara\textcolor{BrickRed}{,}j\textcolor{BrickRed}{)=}m\textcolor{BrickRed}{)} \\
\mbox{}\ \ \ \ \ \ \ \ \textcolor{Red}{\}} \\
\mbox{}\ \ \ \ \ \ \ \ \textbf{\textcolor{Blue}{else}} \\
\mbox{}\ \ \ \ \ \ \ \ \ \ \ \ mejora\textcolor{BrickRed}{=}\textbf{\textcolor{Blue}{false}} \\
\mbox{}\ \ \ \ \textcolor{Red}{\}}\textbf{\textcolor{Blue}{while}}\ \textcolor{BrickRed}{!}non$\_$selected\textcolor{BrickRed}{.}empty\ \textbf{\textcolor{Blue}{and}}\ mejora \\
\mbox{} \\
\mbox{}\ \ \ \ \textbf{\textcolor{Blue}{return}}\ mascara \\
\mbox{}\textcolor{Red}{\}} \\
\mbox{}
}
  \end{itemize}


\item \textbf{Algoritmo Búsqueda Local (BL)} empleado:
  
  El operador de generación de vecino usado ha sido:\\
  
  \small\texttt{% Generator: GNU source-highlight, by Lorenzo Bettini, http://www.gnu.org/software/src-highlite
\noindent
\mbox{}\textbf{\textcolor{Blue}{def}}\ flip\textcolor{BrickRed}{(}mask\textcolor{BrickRed}{,}j\textcolor{BrickRed}{)}\textcolor{Red}{\{} \\
\mbox{}\ \ \ \ mask\textcolor{BrickRed}{[}j\textcolor{BrickRed}{]}\ \textcolor{BrickRed}{=}\ \textcolor{BrickRed}{(}mask\textcolor{BrickRed}{[}j\textcolor{BrickRed}{]+}\textcolor{Purple}{1}\textcolor{BrickRed}{)\%\%}\textcolor{Purple}{2} \\
\mbox{}\ \ \ \ \textbf{\textcolor{Blue}{return}}\ mask \\
\mbox{}\textcolor{Red}{\}} \\
\mbox{}
}
  
  El algoritmo, propiamente dicho, que se ha empleado ha sido:\\
  
  \small\texttt{% Generator: GNU source-highlight, by Lorenzo Bettini, http://www.gnu.org/software/src-highlite
\noindent
\mbox{}\textbf{\textcolor{Blue}{def}}\ BL\textcolor{BrickRed}{()}\textcolor{Red}{\{} \\
\mbox{}\ \ \ \ \textbf{\textcolor{Blue}{do}} \\
\mbox{}\ \ \ \ \ \ \ \ no$\_$seleccionados\ \textcolor{BrickRed}{=}\ \textcolor{Red}{\{}\textcolor{Purple}{1}\textcolor{BrickRed}{,}\textcolor{Purple}{2}\textcolor{BrickRed}{,...}n\textcolor{Red}{\}} \\
\mbox{}\ \ \ \ \ \ \ \ \textbf{\textcolor{Blue}{do}} \\
\mbox{}\ \ \ \ \ \ \ \ \ \ \ \ i\ \textcolor{BrickRed}{=}\ random\textcolor{BrickRed}{(}no$\_$seleccionados\textcolor{BrickRed}{)} \\
\mbox{}\ \ \ \ \ \ \ \ \ \ \ \ no$\_$seleccionados\ \textcolor{BrickRed}{=}\ no$\_$seleccionados\ \textcolor{BrickRed}{-}\textcolor{Red}{\{}i\textcolor{Red}{\}} \\
\mbox{}\ \ \ \ \ \ \ \ \ \ \ \ evs$\_$hechas\textcolor{BrickRed}{++} \\
\mbox{} \\
\mbox{}\ \ \ \ \ \ \ \ \ \ \ \ \textbf{\textcolor{Blue}{if}}\ \textcolor{BrickRed}{(}tasa$\_$acierto\textcolor{BrickRed}{(}flip\textcolor{BrickRed}{(}mascara\textcolor{BrickRed}{,}\ i\textcolor{BrickRed}{)}\ \textcolor{BrickRed}{\textgreater{}}\ mejor$\_$tasa\textcolor{BrickRed}{)} \\
\mbox{}\ \ \ \ \ \ \ \ \ \ \ \ \ \ \ \ mascara\ \textcolor{BrickRed}{=}\ flip\textcolor{BrickRed}{(}mascara\textcolor{BrickRed}{,}i\textcolor{BrickRed}{)} \\
\mbox{}\ \ \ \ \ \ \ \ \ \ \ \ \ \ \ \ mejor$\_$tasa\ \textcolor{BrickRed}{=}\ tasa$\_$acierto\textcolor{BrickRed}{(}mascara\textcolor{BrickRed}{)} \\
\mbox{}\ \ \ \ \ \ \ \ \ \ \ \ \ \ \ \ mejora$\_$encontrada\ \textcolor{BrickRed}{=}\ \textbf{\textcolor{Blue}{true}} \\
\mbox{}\ \ \ \ \ \ \ \ \textbf{\textcolor{Blue}{while}}\ \textcolor{BrickRed}{!}no$\_$seleccionados\textcolor{BrickRed}{.}empty\ \textbf{\textcolor{Blue}{and}}\ \textcolor{BrickRed}{!}mejora$\_$encontrada \\
\mbox{}\ \ \ \ \textbf{\textcolor{Blue}{while}}\ evs$\_$hechas\ \textcolor{BrickRed}{\textless{}}\ \ tope$\_$evs\ \textbf{\textcolor{Blue}{and}}\ mejora$\_$encontrada \\
\mbox{} \\
\mbox{}\ \ \textbf{\textcolor{Blue}{return}}\ mascara \\
\mbox{}\textcolor{Red}{\}} \\
\mbox{}
}

\item Criterio de aceptación de solución: se considera una función mejor cuando aumenta la función objetivo (tasa de 
clasificación del 3-knn para la máscara dada por esa solución).
  
\item Número de multiarranques: se empleará la búsqueda local sobre 25 máscaras en el BMB, GRASP e ILS.

\end{itemize}

\newpage
\section{Profundización en los algoritmos}
Se incluyen a continuación una descripción de cada metaheurística empleada junto a un pseudocódigo de las mismas.

Sea a partir de aquí $n$ el número de predictores de los respectivos \textit{datasets}.
\subsection{Búsqueda multiarranque básica}

\small{\texttt{% Generator: GNU source-highlight, by Lorenzo Bettini, http://www.gnu.org/software/src-highlite
\noindent
\mbox{}\textbf{\textcolor{Blue}{def}}\ BMB\textcolor{BrickRed}{()}\textcolor{Red}{\{} \\
\mbox{}\ \ \ \ \textit{\textcolor{Brown}{\#\ n\ el\ número\ de\ predictores\ de\ cada\ dataset}} \\
\mbox{}\ \ \ \ mascaras\ \textcolor{BrickRed}{=}\ random$\_$init\textcolor{BrickRed}{(}n\textcolor{BrickRed}{)} \\
\mbox{}\ \ \ \ mascaras\ \textcolor{BrickRed}{=}\ \textcolor{BrickRed}{[}\ BL\textcolor{BrickRed}{(}m\textcolor{BrickRed}{):}\ m\ \textbf{\textcolor{Blue}{in}}\ mascaras\ \textcolor{BrickRed}{]} \\
\mbox{}\ \ \ \ tasas\ \textcolor{BrickRed}{=}\ \textcolor{BrickRed}{[}\ tasa\textcolor{BrickRed}{.}clas\textcolor{BrickRed}{(}m\textcolor{BrickRed}{):}\ m\ \textbf{\textcolor{Blue}{in}}\ mascaras\ \textcolor{BrickRed}{]} \\
\mbox{} \\
\mbox{}\ \ \ \ \textbf{\textcolor{Blue}{return}}\ mascaras\ \textcolor{BrickRed}{[}arg\ max\ tasas\textcolor{BrickRed}{]} \\
\mbox{}\textcolor{Red}{\}} \\
\mbox{}
}}
\normalsize

\subsection{GRASP}

\small{\texttt{% Generator: GNU source-highlight, by Lorenzo Bettini, http://www.gnu.org/software/src-highlite
\noindent
\mbox{}\textbf{\textcolor{Blue}{def}}\ GRASP\textcolor{BrickRed}{()}\textcolor{Red}{\{} \\
\mbox{}\ \ \ \ mascaras\ \textcolor{BrickRed}{=}\ \textcolor{BrickRed}{[}\ random$\_$greedy\textcolor{BrickRed}{(}n\textcolor{BrickRed}{)}\ from\ \textcolor{Purple}{1}\ to\ max$\_$arranques\ \textcolor{BrickRed}{]} \\
\mbox{}\ \ \ \ mascaras\ \textcolor{BrickRed}{=}\ \textcolor{BrickRed}{[}\ BL\textcolor{BrickRed}{(}m\textcolor{BrickRed}{):}\ m \textbf{\textcolor{Blue}{$\in$}}\ mascaras\ \textcolor{BrickRed}{]} \\
\mbox{}\ \ \ \ tasas\ \textcolor{BrickRed}{=}\ \textcolor{BrickRed}{[}\ tasa\textcolor{BrickRed}{.}clas\textcolor{BrickRed}{(}m\textcolor{BrickRed}{):}\ m \textbf{\textcolor{Blue}{$\in$}}\ mascaras\ \textcolor{BrickRed}{]} \\
\mbox{} \\
\mbox{}\ \ \ \ \textbf{\textcolor{Blue}{return}}\ mascaras\ \textcolor{BrickRed}{[}arg\ max\ tasas\textcolor{BrickRed}{]} \\
\mbox{}\textcolor{Red}{\}} \\
\mbox{}
}}
\normalsize

\subsection{Búsqueda Local Reiterada (ILS)}

\small{\texttt{% Generator: GNU source-highlight, by Lorenzo Bettini, http://www.gnu.org/software/src-highlite
\noindent
\mbox{}\textbf{\textcolor{Blue}{def}}\ ILS\textcolor{BrickRed}{()}\textcolor{Red}{\{} \\
\mbox{}\ \ \ \ mejor$\_$mascara\ \textcolor{BrickRed}{=}\ \textcolor{BrickRed}{[}\ random\textcolor{BrickRed}{(}\textcolor{Red}{\{}\textcolor{Purple}{1}\textcolor{BrickRed}{,}\textcolor{Purple}{0}\textcolor{Red}{\}}\textcolor{BrickRed}{)}\ from\ \textcolor{Purple}{0}\ to\ n\ \textcolor{BrickRed}{]} \\
\mbox{}\ \ \ \ num$\_$mutaciones\ \textcolor{BrickRed}{=}\ prob$\_$mutacion\ \textcolor{BrickRed}{*}\ n \\
\mbox{} \\
\mbox{}\ \ \ \ \textbf{\textcolor{Blue}{while}}\textcolor{BrickRed}{(}n$\_$eval\ \textcolor{BrickRed}{\textless{}}\ max\textcolor{BrickRed}{.}arranques\textcolor{BrickRed}{)}\textcolor{Red}{\{} \\
\mbox{}\ \ \ \ \ \ \ \ mascara\ \textcolor{BrickRed}{\textless{}-}\ BL\textcolor{BrickRed}{(}mascara\textcolor{BrickRed}{)} \\
\mbox{} \\
\mbox{}\ \ \ \ \ \ \ \ \textbf{\textcolor{Blue}{if}}\ \textcolor{BrickRed}{(}tasa\textcolor{BrickRed}{(}mascara\textcolor{BrickRed}{)}\ \textcolor{BrickRed}{\textgreater{}}\ tasa\textcolor{BrickRed}{(}mejor$\_$mascara\textcolor{BrickRed}{))}\textcolor{Red}{\{} \\
\mbox{}\ \ \ \ \ \ \ \ \ \ \ \ mejor$\_$mascara\ \textcolor{BrickRed}{=}\ mascara \\
\mbox{}\ \ \ \ \ \ \ \ \textcolor{Red}{\}} \\
\mbox{} \\
\mbox{}\ \ \ \ \ \ \ \ mutaciones\ \textcolor{BrickRed}{=}\ \textcolor{BrickRed}{[}random\textcolor{BrickRed}{(}\textcolor{Red}{\{}\textcolor{Purple}{1}\textcolor{BrickRed}{...}n\textcolor{Red}{\}}\textcolor{BrickRed}{)}\ from\ \textcolor{Purple}{0}\ to\ num$\_$mutaciones\ \textcolor{BrickRed}{]} \\
\mbox{} \\
\mbox{}\ \ \ \ \ \ \ \ foreach\ m\ \textbf{\textcolor{Blue}{in}}\ mutaciones\textcolor{Red}{\{} \\
\mbox{}\ \ \ \ \ \ \ \ \ \ \ \ mascara\ \textcolor{BrickRed}{=}\ flip\textcolor{BrickRed}{(}mascara\textcolor{BrickRed}{,}\ m\textcolor{BrickRed}{)} \\
\mbox{}\ \ \ \ \ \ \ \ \textcolor{Red}{\}} \\
\mbox{}\ \ \ \ \textcolor{Red}{\}} \\
\mbox{}\ \ \ \ \textbf{\textcolor{Blue}{return}}\ mascara \\
\mbox{}\textcolor{Red}{\}} \\
\mbox{}
}}
\normalsize

\section{Algoritmo de comparación: SFS}
\small{\texttt{% Generator: GNU source-highlight, by Lorenzo Bettini, http://www.gnu.org/software/src-highlite
\noindent
\mbox{}\textbf{\textcolor{Blue}{def}}\ SFS\textcolor{BrickRed}{()}\textcolor{Red}{\{} \\
\mbox{}\ \ \ \ non$\_$selected\ \textcolor{BrickRed}{=}\ \textcolor{Red}{\{}\textcolor{Purple}{1}\textcolor{BrickRed}{,...}n\textcolor{Red}{\}} \\
\mbox{}\ \ \ \ mascara\ \textcolor{BrickRed}{=}\ \textcolor{Red}{\{}\textcolor{Purple}{0}\textcolor{BrickRed}{,}\textcolor{Purple}{0}\textcolor{BrickRed}{...}\textcolor{Purple}{0}\textcolor{Red}{\}} \\
\mbox{}\ \ \ \ mejora\ \textcolor{BrickRed}{=}\ \textbf{\textcolor{Blue}{true}} \\
\mbox{} \\
\mbox{}\ \ \ \ \textbf{\textcolor{Blue}{do}}\textcolor{Red}{\{} \\
\mbox{}\ \ \ \ \ \ \ \ j\ \textcolor{BrickRed}{=}\ max\ arg\textcolor{Red}{\{}tasa\textcolor{BrickRed}{(}flip\textcolor{BrickRed}{(}mascara\textcolor{BrickRed}{,}x\textcolor{BrickRed}{)):}\ x\$\textcolor{BrickRed}{\textbackslash{}}\textbf{\textcolor{Blue}{in}}\$\ non$\_$selected\textcolor{Red}{\}} \\
\mbox{} \\
\mbox{}\ \ \ \ \ \ \ \ \textbf{\textcolor{Blue}{if}}\ \textcolor{BrickRed}{(}tasa\textcolor{BrickRed}{(}flip\textcolor{BrickRed}{(}mascara\textcolor{BrickRed}{,}j\textcolor{BrickRed}{))}\ \textcolor{BrickRed}{\textgreater{}}\ tasa\textcolor{BrickRed}{(}mascara\textcolor{BrickRed}{))}\textcolor{Red}{\{} \\
\mbox{}\ \ \ \ \ \ \ \ \ \ \ \ non$\_$selected\ \textcolor{BrickRed}{=}\ non$\_$selected\ \textcolor{BrickRed}{-}\ \textcolor{Red}{\{}j\textcolor{Red}{\}} \\
\mbox{}\ \ \ \ \ \ \ \ \ \ \ \ mascara\ \textcolor{BrickRed}{[}j\textcolor{BrickRed}{]}\ \textcolor{BrickRed}{=}\ \textcolor{Purple}{1} \\
\mbox{}\ \ \ \ \ \ \ \ \textcolor{Red}{\}} \\
\mbox{}\ \ \ \ \ \ \ \ \textbf{\textcolor{Blue}{else}} \\
\mbox{}\ \ \ \ \ \ \ \ \ \ \ \ mejora\ \textcolor{BrickRed}{=}\ \textbf{\textcolor{Blue}{false}} \\
\mbox{}\ \ \ \ \textcolor{Red}{\}}\textbf{\textcolor{Blue}{while}}\ mejora \\
\mbox{} \\
\mbox{}\ \ \textbf{\textcolor{Blue}{return}}\ mascara \\
\mbox{}\textcolor{Red}{\}} \\
\mbox{}
}}

Partiendo de una selección de características vacía (máscara nula), va añadiendo a cada paso la característica
de las no escogidas hasta el momento que maximiza la tasa de clasificación añadiéndola a la solución, hasta
que la característica escogida no mejore a la mejor solución hasta el momento

\section{Implementación de la práctica}
Para la implementación de la práctica, se ha optado, al igual que en la primera práctica de trayectorias simples,
por usar el lenguaje de programación R, reutilizando toda la lectura de ficheros ya implementada para la primera práctica

El clasificador \texttt{3nn} leaving one out empleado es el incluido en el paquete \texttt{class}, de nombre
\texttt{knn.cv}. Para su uso, se le pasa como argumento el número de vecinos a considerar (3). El argumento 
\texttt{use.all=TRUE} indica que en caso de empate considere todas las etiquetas de las instancias con distancias 
iguales a la mayoritaria (en este caso las tres), y escoja la etiqueta mayoritaria de entre todas. 

Ya se comentó en la primera práctica la desventaja de usar este paquete de R en caso de triple empate al escoger la etiqueta
que asignar de manera aleatoria.

Se emplea de nuevo la función (\texttt{normalize(data.frame)}) que limpia los \textit{datasets} de columnas con
un único valor, establece los nombres de los atributos todos a minúsculas, reordena los atributos del dataset para que
la clase sea el último de todos, y hace una transformación con las columnas para que todos los valores estén comprendidos
entre 0 y 1.

Se describe a continuación un esquema de los ficheros de código:
\begin{itemize}
 \item \hrefr{main.r}: de arriba a abajo, carga los paquetes necesarios, incluye el código fuente de otros ficheros
  (\hrefr{aux.r}, \hrefr{NN3.r}, \hrefr{SFS.r}, \hrefr{BL.r},\ldots) y lee los parámetros necesarios, entre ellos las 
  semillas aleatorias (12345678, 23456781, 34567812, 45678123, 56781234) del fichero \hrefr{params.r}
  
  Se ejecuta la función \texttt{cross.eval(algoritmo)}, que almacena los resultados de la ejecución en la lista
  \texttt{algoritmo.results}. Esta lista contendrá 3 dataframes con los datos del 5x2 cross validation, uno por dataset, 
  con 10 filas cada uno  y columnas las tasas de clasificación de test y train, la tasa de reducción y el tiempo de 
  ejecución. Incluye también 3 datasets más con la media de los datos anteriores para cada conjunto de datos.
 
 \item \hrefr{aux.r}: contiene las implementaciones de la función de normalización de datasets, la función de
 particionado de \texttt{data.frames}, la función objetivo \texttt{tasa.clas} y la función \texttt{cross.eval}
 descrita arriba.
 
 \item \hrefr{BL.r}, \hrefr{BMB.r}, \hrefr{GRASP.r}, \hrefr{ILS.r}: contiene la implementación de los algoritmos de Búsqueda Local (BL), Búsqueda 
 Multiarranque Básica(BMB), Búsqueda Multiarranque GRASP(GRASP) y Búsqueda Local Reiterada(ILS), respectivamente. 
 Asimismo también incluye los inicializadores de máscaras aleatorias (\texttt{random.init}), en \texttt{BL.r} 
 y de máscaras greedy aleatorizadas (\texttt{random.greedy.init}), este en \texttt{GRASP.r}
 
 \item \hrefr{params.r}: fichero del que se leen los parámetros:
  \begin{itemize}
    \item \texttt{semilla}: vector de semillas aleatorias para poder reproducir los análisis hechos.
    \item \texttt{BMB.num.sols.init, GRASP.num.sols.init, ILS.num.sols.init}: número de rearranques del algoritmo homónimo.
    Por defecto seteados a 25.
    \item \texttt{GRASP.alpha}: tolerancia para aceptar, en el algoritmo GRASP, una solución en la lista de candidatos 
    respecto a la mejor solución en esa iteración. Por defecto, $\alpha=0.3$
    \item \texttt{ILS.coef.mutacion}: multiplicado por $n$ (número de características en total), es el número de coeficientes
    de una máscara que se mutarán en la Búsqueda Local Reiterada. En nuestro caso, 0.1 por defecto (un 10\%).
  \end{itemize}
 \end{itemize}
 
 Para verificar los datos aportados para el algoritmo \texttt{algx}, a saber, basta ejecutar todo el fichero \texttt{main.r}
 hasta justo antes de la sección \textit{Obtención de resultados}. A continuación, si se ejecuta la línea 
 \texttt{algx.results <- cross.eval(algx)} se obtienen almacenados en una lista los valores de tasas y tiempos de ejecución
 aportados en la presente memoria. 
 
 Para trabajar correctamente, el working path debe estar seteado a la carpeta de códigos fuente. Se puede consultar el 
 valor actual mediante \texttt{getwd()} y setearlo mediante \texttt{setwd(path)}
 
 Cuando se cambia algo en algún algoritmo o en \texttt{params.r} hay que recargar en \texttt{main.r}
 la línea del \texttt{source(file=...)} correspondiente.
 
 \section{Experimentos y análisis de resultados}
 \subsection{Tablas de resultados}
 \begin{table}[H]	
    \caption*{Resultados del 3NN}
    \begin{adjustbox}{width=1.1\textwidth}
    \begin{tabular}{|c|r|r|r|r|r|r|r|r|r|r|r|r|}
    \hline
    \multicolumn{1}{|l|}{} & \multicolumn{ 4}{c|}{\textbf{\textit{Wdbc}}} & \multicolumn{ 4}{c|}{\textbf{\textit{Movement\_Libras}}} & \multicolumn{ 4}{c|}{\textbf{\textit{Arrhythmia}}} \\ \hline
    & \multicolumn{1}{c|}{\textbf{\textit{\%\_clas test}}} & \multicolumn{1}{c|}{\textbf{\textit{\%\_clas train}}} & \multicolumn{1}{c|}{\textbf{\textit{tasa\_red}}} & \multicolumn{1}{c|}{\textbf{T(s)}} & \multicolumn{1}{c|}{\textbf{\textit{\%\_clas test}}} & \multicolumn{1}{c|}{\textbf{\textit{\%\_clas train}}} & \multicolumn{1}{c|}{\textbf{\textit{tasa\_red}}} & \multicolumn{1}{c|}{\textbf{T(s)}} & \multicolumn{1}{c|}{\textbf{\textit{\%\_clas test}}} & \multicolumn{1}{c|}{\textbf{\textit{\%\_clas train}}} & \multicolumn{1}{c|}{\textbf{\textit{tasa\_red}}} & \multicolumn{1}{c|}{\textbf{T(s)}} \\ \hline
    \textbf{Partición 1-1} & 95.78947 & 97.53521 & 0.00000 & 0.00000 & 65.00000 & 66.66667 & 0.00000 & 0.00000 & 65.46392 & 65.62500 & 0.00000 & 0.00000 \\ \hline
    \textbf{Partición 1-2} & 97.53521 & 95.78947 & 0.00000 & 0.00000 & 66.11111 & 67.77778 & 0.00000 & 0.00000 & 65.62500 & 65.97938 & 0.00000 & 0.00000 \\ \hline
    \textbf{Partición 2-1} & 97.19298 & 95.77465 & 0.00000 & 0.00000 & 72.22222 & 64.44444 & 0.00000 & 0.00000 & 62.88660 & 61.45833 & 0.00000 & 0.00000 \\ \hline
    \textbf{Partición 2-2} & 95.77465 & 97.19298 & 0.00000 & 0.00000 & 63.33333 & 70.00000 & 0.00000 & 0.00000 & 63.02083 & 63.91753 & 0.00000 & 0.00000 \\ \hline
    \textbf{Partición 3-1} & 96.14035 & 96.47887 & 0.00000 & 0.00000 & 72.77778 & 65.00000 & 0.00000 & 0.00000 & 62.37113 & 64.06250 & 0.00000 & 0.00000 \\ \hline
    \textbf{Partición 3-2} & 96.47887 & 96.14035 & 0.00000 & 0.00000 & 63.33333 & 75.00000 & 0.00000 & 0.00000 & 63.54167 & 62.88660 & 0.00000 & 0.00000 \\ \hline
    \textbf{Partición 4-1} & 95.43860 & 97.88732 & 0.00000 & 0.00000 & 74.44444 & 66.66667 & 0.00000 & 0.00000 & 64.94845 & 62.50000 & 0.00000 & 0.00000 \\ \hline
    \textbf{Partición 4-2} & 97.88732 & 95.43860 & 0.00000 & 0.00000 & 64.44444 & 72.77778 & 0.00000 & 0.00000 & 61.45833 & 62.88660 & 0.00000 & 0.00000 \\ \hline
    \textbf{Partición 5-1} & 96.49123 & 96.83099 & 0.00000 & 0.00000 & 63.33333 & 68.33333 & 0.00000 & 0.00000 & 61.85567 & 61.45833 & 0.00000 & 0.00000 \\ \hline
    \textbf{Partición 5-2} & 96.83099 & 96.49123 & 0.00000 & 0.00000 & 67.77778 & 65.55556 & 0.00000 & 0.00000 & 60.41667 & 62.37113 & 0.00000 & 0.00000 \\ \hline
    \textbf{Media} & 96.55597 & 96.55597 & 0.00000 & 0.00000 & 67.27778 & 68.22222 & 0.00000 & 0.00000 & 63.15883 & 63.31454 & 0.00000 & 0.00000 \\ \hline
    \end{tabular}
    \end{adjustbox}
    \label{NN3}
  \end{table}
  
   \begin{table}[H]	
    \caption*{Resultados del SFS}
    \begin{adjustbox}{width=1.1\textwidth}
   \begin{tabular}{|c|r|r|r|r|r|r|r|r|r|r|r|r|}
    \hline
    \multicolumn{1}{|l|}{} & \multicolumn{ 4}{c|}{\textbf{\textit{Wdbc}}} & \multicolumn{ 4}{c|}{\textbf{\textit{Movement\_Libras}}} & \multicolumn{ 4}{c|}{\textbf{\textit{Arrhytmia}}} \\ \hline
    \multicolumn{1}{|l|}{} & \multicolumn{1}{c|}{\textbf{\textit{\%\_clas\_test}}} & \multicolumn{1}{c|}{\textbf{\textit{\%\_clas train}}} & \multicolumn{1}{c|}{\textbf{\textit{tasa\_red}}} & \multicolumn{1}{c|}{\textbf{T(s)}} & \multicolumn{1}{c|}{\textbf{\textit{\%\_clas\_test}}} & \multicolumn{1}{c|}{\textbf{\textit{\%\_clas\_train}}} & \multicolumn{1}{c|}{\textbf{\textit{tasa\_red}}} & \multicolumn{1}{c|}{\textbf{T(s)}} & \multicolumn{1}{c|}{\textbf{\textit{\%\_clas\_test}}} & \multicolumn{1}{c|}{\textbf{\textit{\%\_clas\_train}}} & \multicolumn{1}{c|}{\textbf{\textit{tasa\_red}}} & \multicolumn{1}{c|}{\textbf{T(s)}} \\ \hline
    \textbf{Partición 1-1} & 91.92982 & 97.53521 & 0.86667 & 0.15400 & 63.88889 & 70.55556 & 0.88889 & 1.01200 & 64.94845 & 75.00000 & 0.98419 & 1.84000 \\ \hline
    \textbf{Partición 1-2} & 95.77465 & 97.89474 & 0.80000 & 0.25600 & 65.00000 & 68.33333 & 0.87778 & 1.29800 & 75.00000 & 78.35052 & 0.98419 & 1.98900 \\ \hline
    \textbf{Partición 2-1} & 97.19298 & 97.88732 & 0.86667 & 0.15200 & 64.44444 & 66.66667 & 0.93333 & 0.57300 & 64.94845 & 76.56250 & 0.98419 & 1.95300 \\ \hline
    \textbf{Partición 2-2} & 94.01408 & 97.19298 & 0.80000 & 0.26600 & 62.22222 & 69.44444 & 0.87778 & 1.13400 & 73.43750 & 71.13402 & 0.98419 & 2.22600 \\ \hline
    \textbf{Partición 3-1} & 94.38596 & 95.77465 & 0.93333 & 0.09700 & 66.66667 & 66.11111 & 0.88889 & 1.02800 & 72.16495 & 80.20833 & 0.98024 & 2.34900 \\ \hline
    \textbf{Partición 3-2} & 91.54930 & 96.14035 & 0.90000 & 0.14500 & 62.22222 & 74.44444 & 0.88889 & 1.20200 & 67.70833 & 74.22680 & 0.98024 & 2.20000 \\ \hline
    \textbf{Partición 4-1} & 94.03509 & 97.88732 & 0.90000 & 0.12000 & 71.66667 & 72.22222 & 0.88889 & 1.31100 & 74.22680 & 76.04167 & 0.98024 & 2.02000 \\ \hline
    \textbf{Partición 4-2} & 92.25352 & 94.38596 & 0.90000 & 0.11700 & 65.55556 & 77.22222 & 0.90000 & 0.87400 & 67.70833 & 76.80412 & 0.98419 & 2.44800 \\ \hline
    \textbf{Partición 5-1} & 94.73684 & 95.77465 & 0.86667 & 0.15500 & 58.33333 & 75.55556 & 0.91111 & 0.76600 & 67.52577 & 75.52083 & 0.98419 & 2.30900 \\ \hline
    \textbf{Partición 5-2} & 90.49296 & 96.49123 & 0.86667 & 0.15300 & 65.00000 & 67.22222 & 0.90000 & 0.99900 & 70.83333 & 74.74227 & 0.98814 & 1.33000 \\ \hline
    \textbf{Media} & 93.63652 & 96.69644 & 0.87000 & 0.16150 & 64.50000 & 70.77778 & 0.89556 & 1.01970 & 69.85019 & 75.85911 & 0.98340 & 2.06640 \\ \hline
    \end{tabular}
    \end{adjustbox}
    \label{SFS}
  \end{table}
  
  \begin{table}[H]
  \caption*{Resultados del BMB}
  \begin{adjustbox}{width=1.1\textwidth}
  \begin{tabular}{|c|r|r|r|r|r|r|r|r|r|r|r|r|}
  \hline
  \multicolumn{1}{|l|}{} & \multicolumn{ 4}{c|}{\textbf{\textit{Wdbc}}} & \multicolumn{ 4}{c|}{\textbf{\textit{Movement\_Libras}}} & \multicolumn{ 4}{c|}{\textbf{\textit{Arrhytmia}}} \\ \hline
  \multicolumn{1}{|l|}{} & \multicolumn{1}{c|}{\textbf{\textit{\%\_clas\_test}}} & \multicolumn{1}{c|}{\textbf{\textit{\%\_clas train}}} & \multicolumn{1}{c|}{\textbf{\textit{tasa\_red}}} & \multicolumn{1}{c|}{\textbf{T(s)}} & \multicolumn{1}{c|}{\textbf{\textit{\%\_clas\_test}}} & \multicolumn{1}{c|}{\textbf{\textit{\%\_clas\_train}}} & \multicolumn{1}{c|}{\textbf{\textit{tasa\_red}}} & \multicolumn{1}{c|}{\textbf{T(s)}} & \multicolumn{1}{c|}{\textbf{\textit{\%\_clas\_test}}} & \multicolumn{1}{c|}{\textbf{\textit{\%\_clas\_train}}} & \multicolumn{1}{c|}{\textbf{\textit{tasa\_red}}} & \multicolumn{1}{c|}{\textbf{T(s)}} \\ \hline
  \textbf{Partición 1-1} & 94.03509 & 98.94366 & 0.40000 & 4.89100 & 62.77778 & 72.22222 & 0.58889 & 24.48300 & 66.49485 & 68.75000 & 0.47431 & 559.55700 \\ \hline
  \textbf{Partición 1-2} & 96.12676 & 98.24561 & 0.36667 & 5.21200 & 67.22222 & 64.44444 & 0.48889 & 24.28400 & 65.10417 & 71.64948 & 0.58893 & 551.05100 \\ \hline
  \textbf{Partición 2-1} & 95.43860 & 97.53521 & 0.53333 & 6.58300 & 68.88889 & 72.22222 & 0.55556 & 23.17700 & 62.88660 & 69.79167 & 0.56917 & 459.46600 \\ \hline
  \textbf{Partición 2-2} & 94.01408 & 98.94737 & 0.56667 & 6.90500 & 62.77778 & 76.11111 & 0.46667 & 21.58600 & 64.58333 & 66.49485 & 0.49802 & 472.28700 \\ \hline
  \textbf{Partición 3-1} & 96.14035 & 97.88732 & 0.50000 & 5.03300 & 71.66667 & 70.55556 & 0.51111 & 21.15200 & 64.43299 & 70.83333 & 0.45850 & 636.12600 \\ \hline
  \textbf{Partición 3-2} & 94.36620 & 98.24561 & 0.46667 & 5.36500 & 66.66667 & 79.44444 & 0.42222 & 23.49600 & 63.54167 & 68.04124 & 0.56522 & 454.29100 \\ \hline
  \textbf{Partición 4-1} & 94.73684 & 98.94366 & 0.46667 & 4.78800 & 68.33333 & 68.88889 & 0.60000 & 21.06900 & 64.94845 & 67.18750 & 0.50988 & 539.35300 \\ \hline
  \textbf{Partición 4-2} & 95.42254 & 97.54386 & 0.50000 & 4.91600 & 66.66667 & 74.44444 & 0.47778 & 20.57700 & 65.10417 & 70.61856 & 0.54941 & 480.48400 \\ \hline
  \textbf{Partición 5-1} & 95.43860 & 97.88732 & 0.46667 & 5.01900 & 66.11111 & 72.22222 & 0.51111 & 19.70900 & 62.37113 & 65.62500 & 0.54941 & 507.68300 \\ \hline
  \textbf{Partición 5-2} & 95.07042 & 98.24561 & 0.46667 & 4.56000 & 65.55556 & 66.11111 & 0.51111 & 21.76900 & 59.89583 & 68.04124 & 0.57312 & 509.55800 \\ \hline
  \textbf{Media} & 95.07895 & 98.24252 & 0.47333 & 5.32720 & 66.66667 & 71.66667 & 0.51333 & 22.13020 & 63.93632 & 68.70329 & 0.53360 & 516.98560 \\ \hline
  \end{tabular}
  \end{adjustbox}
  \label{}
  \end{table}

  \begin{table}[H]
  \caption*{Resultados del GRASP}
  \begin{adjustbox}{width=1.1\textwidth}
  \begin{tabular}{|c|r|r|r|r|r|r|r|r|r|r|r|r|}
  \hline
  \multicolumn{1}{|l|}{} & \multicolumn{ 4}{c|}{\textbf{\textit{Wdbc}}} & \multicolumn{ 4}{c|}{\textbf{\textit{Movement\_Libras}}} & \multicolumn{ 4}{c|}{\textbf{\textit{Arrhytmia}}} \\ \hline
  \multicolumn{1}{|l|}{} & \multicolumn{1}{c|}{\textbf{\textit{\%\_clas\_test}}} & \multicolumn{1}{c|}{\textbf{\textit{\%\_clas train}}} & \multicolumn{1}{c|}{\textbf{\textit{tasa\_red}}} & \multicolumn{1}{c|}{\textbf{T(s)}} & \multicolumn{1}{c|}{\textbf{\textit{\%\_clas\_test}}} & \multicolumn{1}{c|}{\textbf{\textit{\%\_clas\_train}}} & \multicolumn{1}{c|}{\textbf{\textit{tasa\_red}}} & \multicolumn{1}{c|}{\textbf{T(s)}} & \multicolumn{1}{c|}{\textbf{\textit{\%\_clas\_test}}} & \multicolumn{1}{c|}{\textbf{\textit{\%\_clas\_train}}} & \multicolumn{1}{c|}{\textbf{\textit{tasa\_red}}} & \multicolumn{1}{c|}{\textbf{T(s)}} \\ \hline
  \textbf{Partición 1-1} & 95.08772 & 98.94366 & 0.73333 & 6.39000 & 59.44444 & 72.77778 & 0.86667 & 31.34700 & 69.07216 & 77.60417 & 0.94071 & 66.54200 \\ \hline
  \textbf{Partición 1-2} & 94.36620 & 98.59649 & 0.66667 & 7.65700 & 65.00000 & 74.44444 & 0.83333 & 27.10800 & 70.83333 & 81.95876 & 0.93281 & 86.96500 \\ \hline
  \textbf{Partición 2-1} & 95.08772 & 98.23944 & 0.76667 & 6.24400 & 62.22222 & 71.66667 & 0.86667 & 27.14100 & 69.07216 & 80.72917 & 0.94862 & 84.36400 \\ \hline
  \textbf{Partición 2-2} & 96.47887 & 98.94737 & 0.66667 & 7.48500 & 68.88889 & 78.33333 & 0.81111 & 30.23100 & 71.35417 & 78.35052 & 0.95257 & 80.98800 \\ \hline
  \textbf{Partición 3-1} & 96.49123 & 98.23944 & 0.80000 & 6.38600 & 71.11111 & 73.33333 & 0.86667 & 26.74900 & 67.01031 & 82.29167 & 0.90119 & 93.01400 \\ \hline
  \textbf{Partición 3-2} & 95.42254 & 98.24561 & 0.80000 & 6.26100 & 63.88889 & 77.77778 & 0.84444 & 31.06500 & 75.00000 & 77.31959 & 0.92885 & 64.86900 \\ \hline
  \textbf{Partición 4-1} & 95.08772 & 99.29577 & 0.80000 & 7.12200 & 72.77778 & 72.77778 & 0.87778 & 30.86800 & 73.19588 & 77.60417 & 0.92490 & 71.93300 \\ \hline
  \textbf{Partición 4-2} & 98.23944 & 97.89474 & 0.70000 & 6.32900 & 66.11111 & 77.22222 & 0.85556 & 26.87700 & 66.14583 & 79.89691 & 0.95652 & 73.92300 \\ \hline
  \textbf{Partición 5-1} & 94.38596 & 97.88732 & 0.73333 & 6.93500 & 65.00000 & 79.44444 & 0.82222 & 29.86400 & 72.68041 & 83.33333 & 0.91304 & 78.73700 \\ \hline
  \textbf{Partición 5-2} & 92.60563 & 98.59649 & 0.66667 & 6.43600 & 73.88889 & 69.44444 & 0.83333 & 26.30000 & 73.95833 & 81.95876 & 0.91700 & 85.65900 \\ \hline
  \textbf{Media} & 95.32530 & 98.48863 & 0.73333 & 6.72450 & 66.83333 & 74.72222 & 0.84778 & 28.75500 & 70.83226 & 80.10471 & 0.93162 & 78.69940 \\ \hline
  \end{tabular}
  \end{adjustbox}
  \label{}
  \end{table}

  
  \begin{table}[H]
  \caption*{Resultados del ILS}
  \begin{adjustbox}{width=1.1\textwidth}
  \begin{tabular}{|c|r|r|r|r|r|r|r|r|r|r|r|r|}
  \hline
  \multicolumn{1}{|l|}{} & \multicolumn{ 4}{c|}{\textbf{\textit{Wdbc}}} & \multicolumn{ 4}{c|}{\textbf{\textit{Movement\_Libras}}} & \multicolumn{ 4}{c|}{\textbf{\textit{Arrhytmia}}} \\ \hline
  \multicolumn{1}{|l|}{} & \multicolumn{1}{c|}{\textbf{\textit{\%\_clas\_test}}} & \multicolumn{1}{c|}{\textbf{\textit{\%\_clas train}}} & \multicolumn{1}{c|}{\textbf{\textit{tasa\_red}}} & \multicolumn{1}{c|}{\textbf{T(s)}} & \multicolumn{1}{c|}{\textbf{\textit{\%\_clas\_test}}} & \multicolumn{1}{c|}{\textbf{\textit{\%\_clas\_train}}} & \multicolumn{1}{c|}{\textbf{\textit{tasa\_red}}} & \multicolumn{1}{c|}{\textbf{T(s)}} & \multicolumn{1}{c|}{\textbf{\textit{\%\_clas\_test}}} & \multicolumn{1}{c|}{\textbf{\textit{\%\_clas\_train}}} & \multicolumn{1}{c|}{\textbf{\textit{tasa\_red}}} & \multicolumn{1}{c|}{\textbf{T(s)}} \\ \hline
  \textbf{Partición 1-1} & 95.78947 & 99.29577 & 0.66667 & 3.89100 & 65.00000 & 73.33333 & 0.54444 & 17.71700 & 65.97938 & 68.22917 & 0.49407 & 430.85800 \\ \hline
  \textbf{Partición 1-2} & 96.47887 & 98.59649 & 0.63333 & 4.01700 & 68.33333 & 68.88889 & 0.51111 & 19.27500 & 66.14583 & 71.13402 & 0.53360 & 462.95000 \\ \hline
  \textbf{Partición 2-1} & 96.84211 & 98.23944 & 0.60000 & 4.31800 & 70.00000 & 71.11111 & 0.50000 & 24.34100 & 65.46392 & 72.39583 & 0.53360 & 532.44900 \\ \hline
  \textbf{Partición 2-2} & 95.42254 & 98.94737 & 0.50000 & 4.45000 & 65.00000 & 74.44444 & 0.42222 & 22.60300 & 66.14583 & 69.07216 & 0.53360 & 425.70100 \\ \hline
  \textbf{Partición 3-1} & 96.49123 & 97.88732 & 0.50000 & 5.16800 & 72.77778 & 73.33333 & 0.51111 & 22.46600 & 59.27835 & 67.70833 & 0.45059 & 573.51300 \\ \hline
  \textbf{Partición 3-2} & 95.07042 & 98.59649 & 0.40000 & 4.67000 & 65.55556 & 78.33333 & 0.47778 & 20.18200 & 67.18750 & 63.91753 & 0.55731 & 484.83600 \\ \hline
  \textbf{Partición 4-1} & 94.73684 & 99.29577 & 0.43333 & 4.56700 & 71.11111 & 67.22222 & 0.56667 & 19.24300 & 67.01031 & 70.31250 & 0.45059 & 472.73000 \\ \hline
  \textbf{Partición 4-2} & 98.23944 & 98.24561 & 0.66667 & 3.76200 & 67.22222 & 75.00000 & 0.50000 & 18.53900 & 63.02083 & 67.52577 & 0.48617 & 445.80500 \\ \hline
  \textbf{Partición 5-1} & 96.14035 & 98.23944 & 0.60000 & 4.72700 & 63.33333 & 75.00000 & 0.46667 & 23.46200 & 61.34021 & 70.31250 & 0.51779 & 478.56200 \\ \hline
  \textbf{Partición 5-2} & 92.60563 & 98.24561 & 0.43333 & 4.12600 & 69.44444 & 67.77778 & 0.51111 & 20.80600 & 64.58333 & 66.49485 & 0.45059 & 474.51200 \\ \hline
  \textbf{Media} & 95.78169 & 98.55893 & 0.54333 & 4.36960 & 67.77778 & 72.44444 & 0.50111 & 20.86340 & 64.61555 & 68.71027 & 0.50079 & 478.19160 \\ \hline
  \end{tabular}
  \end{adjustbox}
  \label{}
  \end{table}
  
  \subsection{Wdbc}
  
  Es un problema de clasificación binaria en un dataset de 30 predictores y 569 instancias.
  
  \imagen{../data/multi-restart/wdbc.png}{Tasas de clasificación en Wdbc}{wdbcgraph}{0.7}
  
  %\imgn{../data/wdbc3nn.png}{Selección de características del SFS}{1}
  
  %\imgn{../data/wdbctabu.png}{Boxplot de variables del wdbc}{1}
 
  \subsection{Movement libras}
  
  Se trata de un problema de clasficación multiclase (15 clases después de normalizar el dataset). 
  Contiene 360 instancias y 90 predictores.
  
  \imagen{../data/multi-restart/mlibras.png}{Tasas de clasificación en Movement Libras}{wdbcgraph}{0.7}
  
  
  \subsection{Arrhythmia}
  
  Se trata de un problema de clasficación multiclase (5 clases después de normalizar el dataset). 
  Tiene 386 instancias y 253 predictores.
  
  \imagen{../data/multi-restart/arrhythmia.png}{Tasas de clasificación en Arrhythmia}{wdbcgraph}{0.7}
  
  \subsection{Notas finales}
  
  
\end{document}