\documentclass[a4paper,11pt]{article}
% Símbolo del euro
\usepackage[gen]{eurosym}
% Codificación
\usepackage[utf8]{inputenc}
% Idioma
\usepackage[spanish]{babel} % English language/hyphenation
\selectlanguage{spanish}
% Hay que pelearse con babel-spanish para el alineamiento del punto decimal
\decimalpoint
\usepackage{dcolumn}
\newcolumntype{d}[1]{D{.}{\esperiod}{#1}}
\makeatletter
\addto\shorthandsspanish{\let\esperiod\es@period@code}
\makeatother

\usepackage[usenames,dvipsnames]{color} % Coloring code

\usepackage{csvsimple}
\usepackage{adjustbox}
\newsavebox\ltmcbox


% Para matrices
\usepackage{amsmath}

% Símbolos matemáticos
\usepackage{amssymb}
\let\oldemptyset\emptyset
\let\emptyset\varnothing

% Hipervínculos
\usepackage{url}

\usepackage[section]{placeins} % Para gráficas en su sección.
\usepackage{mathpazo} % Use the Palatino font
\usepackage[T1]{fontenc} % Required for accented characters
\newenvironment{allintypewriter}{\ttfamily}{\par}
\setlength{\parindent}{0pt}
\parskip=8pt
\linespread{1.05} % Change line spacing here, Palatino benefits from a slight increase by default


% Imágenes
\usepackage{graphicx}
\usepackage{float}
\usepackage{caption}
\usepackage{wrapfig} % Allows in-line images



% Márgenes
\usepackage{geometry}
 \geometry{
 a4paper,
 total={210mm,297mm},
 left=30mm,
 right=30mm,
 top=25mm,
 bottom=25mm,
 }


% Referencias
\usepackage{fncylab}
\labelformat{figure}{\textit{\figurename\space #1}}

\usepackage{hyperref}
\hypersetup{
  colorlinks   = true, %Colours links instead of ugly boxes
  urlcolor     = blue, %Colour for external hyperlinks
  linkcolor    = blue, %Colour of internal links
  citecolor   = red %Colour of citations
}


\makeatletter
\renewcommand{\@listI}{\itemsep=0pt} % Reduce the space between items in the itemize and enumerate environments and the bibliography
\newcommand{\imagent}[4]{
  \begin{wrapfigure}{#4}{0.7\textwidth}
    \begin{center}
    \includegraphics[width=0.7\textwidth]{#1}
    \end{center}
    \caption{#3}
    \label{#4}
  \end{wrapfigure}
}


\newcommand{\imagen}[4]{
  \begin{minipage}{\linewidth}
    \centering
    \includegraphics[width=#4\textwidth]{#1}
    \captionof{figure}{#2}
    \label{#3}
  \end{minipage} 
}

\newcommand{\imgn}[3]{
  \begin{minipage}{\linewidth}
    \centering
    \includegraphics[width=#3\textwidth]{#1}
    \captionof{figure}{#2}
  \end{minipage} 
}

% Ejemplo de parámetro: ILS.r
\newcommand{\hrefr}[1]{
\href{../bin/#1}{#1}
}

%Customize enumerate tag
\usepackage{enumitem}
%Sections don't get numbered
%\setcounter{secnumdepth}{0}
\newcommand{\blue}[1]{\textcolor{blue}{#1}}
\begin{document}
%%%%%%%%%%%%%%%%%%%%%%%%%%%%%%%%%%%%%%%%%
% University Assignment Title Page 
% LaTeX Template
% Version 1.0 (27/12/12)
%
% This template has been downloaded from:
% http://www.LaTeXTemplates.com
%
% Original author:
% WikiBooks (http://en.wikibooks.org/wiki/LaTeX/Title_Creation)
% Modified by: NCordon (https://github.com/NCordon)
%
% License:
% CC BY-NC-SA 3.0 (http://creativecommons.org/licenses/by-nc-sa/3.0/)
% 
% Instructions for using this template:
% This title page is capable of being compiled as is. This is not useful for 
% including it in another document. To do this, you have two options: 
%
% 1) Copy/paste everything between \begin{document} and \end{document} 
% starting at \begin{titlepage} and paste this into another LaTeX file where you 
% want your title page.
% OR
% 2) Remove everything outside the \begin{titlepage} and \end{titlepage} and 
% move this file to the same directory as the LaTeX file you wish to add it to. 
% Then add \input{./title_page_1.tex} to your LaTeX file where you want your
% title page.
%
%%%%%%%%%%%%%%%%%%%%%%%%%%%%%%%%%%%%%%%%%
\begin{titlepage}

\newcommand{\HRule}{\rule{\linewidth}{0.5mm}} % Defines a new command for the horizontal lines, change thickness here

\center % Center everything on the page
 
%----------------------------------------------------------------------------------------
%	HEADING SECTIONS
%----------------------------------------------------------------------------------------
\textsc{\LARGE Universidad de Granada}\\[1.5cm]
\textsc{\Large Metaheaurísticas}\\[0.5cm] 

%----------------------------------------------------------------------------------------
%	TITLE SECTION
%----------------------------------------------------------------------------------------
\bigskip
\HRule \\[0.4cm]
{ \huge \bfseries Práctica II}\\[0.4cm] % Title of your document
{ \huge \bfseries Selección de características}\\
\HRule \\[1.5cm]
 
%----------------------------------------------------------------------------------------
%	AUTHOR SECTION
%----------------------------------------------------------------------------------------

\begin{minipage}{\textwidth}
\begin{center} \large
\emph{Búsqueda Multiarranque Básica}\\
\emph{GRASP}\\
\emph{Búsqueda Local Reiterada(ILS)}\\
\end{center}
\end{minipage}

%----------------------------------------------------------------------------------------
%	LOGO SECTION
%----------------------------------------------------------------------------------------

\begin{center}
\includegraphics[width=8cm]{../data/ugr.jpg}
\end{center}
%----------------------------------------------------------------------------------------

\begin{minipage}{\textwidth}
\begin{center} \large
Ignacio Cordón Castillo, 25352973G\\
\url{nachocordon@correo.ugr.es}\\
\ \\
$4^{\circ}$ Doble Grado Matemáticas Informática\\
Grupo Prácticas Viernes
\end{center}
\end{minipage}


\vspace{\fill}% Fill the rest of the page with whitespace
\large\today
\end{titlepage}  

\newpage
\tableofcontents
\newpage
% Examples of inclussion of images
%\imagent{ugr.jpg}{Logo de prueba}{ugr}
%\imagen{ugr.jpg}{Logo de prueba}{ugr2}{size relative to the \textwidth}

\section{Descripción del problema}
Dado un \textit{dataset} de instancias ya clasificadas, con una serie de atributos, se pretende comparar las distintas 
metaheaurísticas disponibles para comprobar cuál produce el conjunto de atributos que sirven para obtener una mayor 
tasa de clasificación (número de instancias bien clasificadas sobre el total) usando un clasificador de instancias.
En otras palabras, si tenemos $p$ atributos o características, el número total de subconjuntos de características que podemos
escoger es $2^p$. Cada uno de esos conjuntos o selecciones de características tendrá una tasa de clasificación asociada para
el clasificador que hemos escogido. Como este problema cuando $p$ es muy grande es inabordable por fuerza bruta, se trata de aplicar metaheurísticas para encontrar la mejor selección de características que
podamos de entre esas $2^p$ posibilidades.

Se considera la tasa de clasificación en el problema usando un clasificador \textit{3-knn} leaving one-out. Para cada 
instancia de un conjunto, toma las 3 más cercanas usando la distancia euclídea entre sus atributos, y etiqueta la nueva
instancia en función de la etiqueta mayoritaria de entre esas tres. Se efectúan 5 particiones al 50\% estratificadas
por clase en \textit{train - test}, de modo que la metaheurística proporcionará un conjunto de atributos para el conjunto de
entrenamiento, y se calculará la tasa de clasificación que produce sobre el conjunto de prueba para el \textit{3-knn} dicho
conjunto de atributos. El proceso se repite intercambiando los conjuntos de entrenamiento y de test. La calidad de la 
metaheaurística/algoritmo se calculará como la media de todas las tasas de clasificación obtenidas sobre el conjunto test
(10 en total). Otras medidas que se tendrán en cuenta a la hora de evaluar la bondad de un algoritmo empleado serán:
\begin{itemize}
 \item Tasa de reducción: Porcentaje de características que ha eliminado la máscara sobre el total que podía 
 seleccionar. Mejor cuanto más alta. En nuestros resultados lo hemos expresado como un tanto por uno.
 \item Tiempo de ejecución: Tiempo en segundos que tarda el algoritmo en devolver un conjunto de características.
 Se busca el menor tiempo de ejecución posible.
\end{itemize}


La tasa de clasificación del clasificador se mide (en tanto por uno) como: $$tasa\: clasificacion = \frac{instancias\: 
bien\: clasificadas}{total\: instancias}$$

La representación usada es la binaria (1 o 0 por característica), donde tenemos un vector de tamaño $n$, con $n$ el número 
de atributos (exceptuando la clase), y la metaheaurística/algoritmo ha seleccionado el atributo si y solo si lo ha marcado 
a 1. A una representación de esta forma, la denominamos máscara: $$ mask =\begin{matrix} (0 & 1\ldots 1 & 1 & 0\ldots) 
\end{matrix}$$

\newpage
\section{Descripción elementos comunes de los algoritmos}
Los algoritmos considerados tienen las siguientes componentes:

\begin{itemize} 
\item \textbf{Esquema de representación:} se usarán máscaras, sucesiones de unos y ceros de tanta longitud como atributos haya,
exceptuando la clase, donde un 1 en la posición $i-$ésima indica que esa característica se ha escogido, y un $0$ que no.

\item \textbf{Función objetivo}: la función a maximizar será la tasa de clasificación explicada arriba usando el clasificador
3-knn con la selección de características indicada por la máscara.\\

\small\texttt{% Generator: GNU source-highlight, by Lorenzo Bettini, http://www.gnu.org/software/src-highlite
\noindent
\mbox{}\textbf{\textcolor{Blue}{def}}\ tasa$\_$clasificacion\ \textcolor{BrickRed}{(}train\textcolor{BrickRed}{,}\ mascara\textcolor{BrickRed}{)}\textcolor{Red}{\{} \\
\mbox{}\ \ train\ \textcolor{BrickRed}{=}\ apply\textcolor{BrickRed}{(}mascara\textcolor{BrickRed}{,}\ train\textcolor{BrickRed}{)} \\
\mbox{}\ \ clases\ \textcolor{BrickRed}{=}\ train\textcolor{BrickRed}{[}\textbf{\textcolor{Blue}{class}}\textcolor{BrickRed}{]} \\
\mbox{}\ \ train\ \textcolor{BrickRed}{=}\ train\textcolor{BrickRed}{[-}\textbf{\textcolor{Blue}{class}}\textcolor{BrickRed}{]} \\
\mbox{} \\
\mbox{}\ \ fit\ \textcolor{BrickRed}{=}\ leave\textcolor{BrickRed}{.}one\textcolor{BrickRed}{.}out\textcolor{BrickRed}{.}3knn\textcolor{BrickRed}{(}train\textcolor{BrickRed}{,}\ clases\textcolor{BrickRed}{)} \\
\mbox{} \\
\mbox{}\ \ \textbf{\textcolor{Blue}{return}}\ \textcolor{BrickRed}{(}\textcolor{Purple}{100}\ \textcolor{BrickRed}{*}\ length\textcolor{BrickRed}{(}which\textcolor{BrickRed}{(}clases\ \textcolor{BrickRed}{==}\ fit\textcolor{BrickRed}{))}\ \textcolor{BrickRed}{/}\ length\textcolor{BrickRed}{(}clases\textcolor{BrickRed}{))} \\
\mbox{}\textcolor{Red}{\}} \\
\mbox{}
}

\item \textbf{Generación de solución inicial: }
  \begin{itemize}
   \item BMB: se generarán iterativamente 25 soluciones aleatorias sobre las que se aplicará búsqueda local
   para refinarlas.\\
   
    \small\texttt{% Generator: GNU source-highlight, by Lorenzo Bettini, http://www.gnu.org/software/src-highlite
\noindent
\mbox{}\textbf{\textcolor{Blue}{def}}\ random$\_$mask\textcolor{BrickRed}{(}n\textcolor{BrickRed}{)}\textcolor{Red}{\{} \\
\mbox{}\ \ \ \ mascaras\ \textcolor{BrickRed}{=}\ \textcolor{BrickRed}{[}\ \textcolor{BrickRed}{[}\ random\textcolor{BrickRed}{(}\textcolor{Red}{\{}\textcolor{Purple}{0}\textcolor{BrickRed}{,}\textcolor{Purple}{1}\textcolor{Red}{\}}\textcolor{BrickRed}{)}\ from\ \textcolor{Purple}{0}\ to\ n\textcolor{BrickRed}{]}\ from\ \textcolor{Purple}{0}\ to\ max$\_$arranques\ \textcolor{BrickRed}{]} \\
\mbox{}\textcolor{Red}{\}} \\
\mbox{}
}
   
   \item GRASP: la solución inicial se generará mediante un \textit{greedy} aleatorizado, y se refinará mediante
   la búsqueda local, quedándonos con la mejor de entre 25 generadas.\\
   
   \small\texttt{% Generator: GNU source-highlight, by Lorenzo Bettini, http://www.gnu.org/software/src-highlite
\noindent
\mbox{}\textbf{\textcolor{Blue}{def}}\ random$\_$greedy\textcolor{BrickRed}{(}n\textcolor{BrickRed}{)}\textcolor{Red}{\{} \\
\mbox{}\ \ \ \ non$\_$selected\ \textcolor{BrickRed}{=}\ \textcolor{Red}{\{}\textcolor{Purple}{1}\textcolor{BrickRed}{,...}n\textcolor{Red}{\}} \\
\mbox{}\ \ \ \ mejor$\_$mascara\ \textcolor{BrickRed}{=}\ \textcolor{Red}{\{}\textcolor{Purple}{0}\textcolor{BrickRed}{,}\textcolor{Purple}{0}\textcolor{BrickRed}{...}\textcolor{Purple}{0}\textcolor{Red}{\}} \\
\mbox{}\ \ \ \ mejora\ \textcolor{BrickRed}{=}\ \textbf{\textcolor{Blue}{true}} \\
\mbox{} \\
\mbox{}\ \ \ \ \textbf{\textcolor{Blue}{do}}\textcolor{Red}{\{} \\
\mbox{}\ \ \ \ \ \ \ \ mascaras\ \textcolor{BrickRed}{=}\ \textcolor{Red}{\{\}} \\
\mbox{}\ \ \ \ \ \ \ \ tasas\ \textcolor{BrickRed}{=}\ \textcolor{Red}{\{\}} \\
\mbox{} \\
\mbox{}\ \ \ \ \ \ \ \ foreach\ j\ \textbf{\textcolor{Blue}{in}}\ non$\_$selected\textcolor{Red}{\{} \\
\mbox{}\ \ \ \ \ \ \ \ \ \ \ \ m\ \textcolor{BrickRed}{=}\ flip\textcolor{BrickRed}{(}mascara\textcolor{BrickRed}{,}j\textcolor{BrickRed}{)} \\
\mbox{}\ \ \ \ \ \ \ \ \ \ \ \ mascaras\textcolor{BrickRed}{.}add\textcolor{BrickRed}{(}m\textcolor{BrickRed}{)} \\
\mbox{}\ \ \ \ \ \ \ \ \ \ \ \ tasas\textcolor{BrickRed}{.}add\textcolor{BrickRed}{(}tasa\textcolor{BrickRed}{(}m\textcolor{BrickRed}{))} \\
\mbox{}\ \ \ \ \ \ \ \ \textcolor{Red}{\}} \\
\mbox{} \\
\mbox{}\ \ \ \ \ \ \ \ umbral\ \textcolor{BrickRed}{=}\ alpha\ \textcolor{BrickRed}{*}\ \textcolor{BrickRed}{(}max\textcolor{BrickRed}{(}tasas\textcolor{BrickRed}{)}\ \textcolor{BrickRed}{-}\ min\textcolor{BrickRed}{(}tasas\textcolor{BrickRed}{))} \\
\mbox{}\ \ \ \ \ \ \ \ mascaras\textcolor{BrickRed}{.}delete\ \textcolor{BrickRed}{(}which\textcolor{BrickRed}{(}max\textcolor{BrickRed}{(}tasas\textcolor{BrickRed}{)}\ \textcolor{BrickRed}{-}\ tasas\ \textcolor{BrickRed}{\textgreater{}}\ umbral\textcolor{BrickRed}{))} \\
\mbox{}\ \ \ \ \ \ \ \ m\ \textcolor{BrickRed}{=}\ mascaras\textcolor{BrickRed}{[}\ random\textcolor{Red}{\{}\textcolor{Purple}{1}\textcolor{BrickRed}{..}n\textcolor{Red}{\}}\ \textcolor{BrickRed}{]} \\
\mbox{} \\
\mbox{}\ \ \ \ \ \ \ \ \textbf{\textcolor{Blue}{if}}\ \textcolor{BrickRed}{(}tasa\textcolor{BrickRed}{(}m\textcolor{BrickRed}{)}\ \textcolor{BrickRed}{\textgreater{}}\ tasa\textcolor{BrickRed}{(}mejor$\_$mascara\textcolor{BrickRed}{))}\textcolor{Red}{\{} \\
\mbox{}\ \ \ \ \ \ \ \ \ \ \ \ mejor$\_$mascara\ \textcolor{BrickRed}{=}\ m \\
\mbox{}\ \ \ \ \ \ \ \ \ \ \ \ non$\_$selected\textcolor{BrickRed}{.}delete\textcolor{BrickRed}{(}j\ \textcolor{BrickRed}{:}\ flip\textcolor{BrickRed}{(}mascara\textcolor{BrickRed}{,}j\textcolor{BrickRed}{)=}m\textcolor{BrickRed}{)} \\
\mbox{}\ \ \ \ \ \ \ \ \textcolor{Red}{\}} \\
\mbox{}\ \ \ \ \ \ \ \ \textbf{\textcolor{Blue}{else}} \\
\mbox{}\ \ \ \ \ \ \ \ \ \ \ \ mejora\textcolor{BrickRed}{=}\textbf{\textcolor{Blue}{false}} \\
\mbox{}\ \ \ \ \textcolor{Red}{\}}\textbf{\textcolor{Blue}{while}}\ \textcolor{BrickRed}{!}non$\_$selected\textcolor{BrickRed}{.}empty\ \textbf{\textcolor{Blue}{and}}\ mejora \\
\mbox{} \\
\mbox{}\ \ \ \ \textbf{\textcolor{Blue}{return}}\ mascara \\
\mbox{}\textcolor{Red}{\}} \\
\mbox{}
}
  \end{itemize}


\item \textbf{Algoritmo Búsqueda Local (BL)} empleado:
  
  El operador de generación de vecino usado ha sido:\\
  
  \small\texttt{% Generator: GNU source-highlight, by Lorenzo Bettini, http://www.gnu.org/software/src-highlite
\noindent
\mbox{}\textbf{\textcolor{Blue}{def}}\ flip\textcolor{BrickRed}{(}mask\textcolor{BrickRed}{,}j\textcolor{BrickRed}{)}\textcolor{Red}{\{} \\
\mbox{}\ \ \ \ mask\textcolor{BrickRed}{[}j\textcolor{BrickRed}{]}\ \textcolor{BrickRed}{=}\ \textcolor{BrickRed}{(}mask\textcolor{BrickRed}{[}j\textcolor{BrickRed}{]+}\textcolor{Purple}{1}\textcolor{BrickRed}{)\%\%}\textcolor{Purple}{2} \\
\mbox{}\ \ \ \ \textbf{\textcolor{Blue}{return}}\ mask \\
\mbox{}\textcolor{Red}{\}} \\
\mbox{}
}
  
  El algoritmo, propiamente dicho, que se ha empleado ha sido:\\
  
  \small\texttt{% Generator: GNU source-highlight, by Lorenzo Bettini, http://www.gnu.org/software/src-highlite
\noindent
\mbox{}\textbf{\textcolor{Blue}{def}}\ BL\textcolor{BrickRed}{()}\textcolor{Red}{\{} \\
\mbox{}\ \ \ \ \textbf{\textcolor{Blue}{do}} \\
\mbox{}\ \ \ \ \ \ \ \ no$\_$seleccionados\ \textcolor{BrickRed}{=}\ \textcolor{Red}{\{}\textcolor{Purple}{1}\textcolor{BrickRed}{,}\textcolor{Purple}{2}\textcolor{BrickRed}{,...}n\textcolor{Red}{\}} \\
\mbox{}\ \ \ \ \ \ \ \ \textbf{\textcolor{Blue}{do}} \\
\mbox{}\ \ \ \ \ \ \ \ \ \ \ \ i\ \textcolor{BrickRed}{=}\ random\textcolor{BrickRed}{(}no$\_$seleccionados\textcolor{BrickRed}{)} \\
\mbox{}\ \ \ \ \ \ \ \ \ \ \ \ no$\_$seleccionados\ \textcolor{BrickRed}{=}\ no$\_$seleccionados\ \textcolor{BrickRed}{-}\textcolor{Red}{\{}i\textcolor{Red}{\}} \\
\mbox{}\ \ \ \ \ \ \ \ \ \ \ \ evs$\_$hechas\textcolor{BrickRed}{++} \\
\mbox{} \\
\mbox{}\ \ \ \ \ \ \ \ \ \ \ \ \textbf{\textcolor{Blue}{if}}\ \textcolor{BrickRed}{(}tasa$\_$acierto\textcolor{BrickRed}{(}flip\textcolor{BrickRed}{(}mascara\textcolor{BrickRed}{,}\ i\textcolor{BrickRed}{)}\ \textcolor{BrickRed}{\textgreater{}}\ mejor$\_$tasa\textcolor{BrickRed}{)} \\
\mbox{}\ \ \ \ \ \ \ \ \ \ \ \ \ \ \ \ mascara\ \textcolor{BrickRed}{=}\ flip\textcolor{BrickRed}{(}mascara\textcolor{BrickRed}{,}i\textcolor{BrickRed}{)} \\
\mbox{}\ \ \ \ \ \ \ \ \ \ \ \ \ \ \ \ mejor$\_$tasa\ \textcolor{BrickRed}{=}\ tasa$\_$acierto\textcolor{BrickRed}{(}mascara\textcolor{BrickRed}{)} \\
\mbox{}\ \ \ \ \ \ \ \ \ \ \ \ \ \ \ \ mejora$\_$encontrada\ \textcolor{BrickRed}{=}\ \textbf{\textcolor{Blue}{true}} \\
\mbox{}\ \ \ \ \ \ \ \ \textbf{\textcolor{Blue}{while}}\ \textcolor{BrickRed}{!}no$\_$seleccionados\textcolor{BrickRed}{.}empty\ \textbf{\textcolor{Blue}{and}}\ \textcolor{BrickRed}{!}mejora$\_$encontrada \\
\mbox{}\ \ \ \ \textbf{\textcolor{Blue}{while}}\ evs$\_$hechas\ \textcolor{BrickRed}{\textless{}}\ \ tope$\_$evs\ \textbf{\textcolor{Blue}{and}}\ mejora$\_$encontrada \\
\mbox{} \\
\mbox{}\ \ \textbf{\textcolor{Blue}{return}}\ mascara \\
\mbox{}\textcolor{Red}{\}} \\
\mbox{}
}

\item Criterio de aceptación de solución: se considera una función mejor cuando aumenta la función objetivo (tasa de 
clasificación del 3-knn para la máscara dada por esa solución).
  
\item Número de multiarranques: se empleará la búsqueda local sobre 25 máscaras en el BMB, GRASP e ILS.

\end{itemize}

\newpage
\section{Profundización en los algoritmos}
Se incluyen a continuación una descripción de cada metaheurística empleada junto a un pseudocódigo de las mismas.

Sea a partir de aquí $n$ el número de predictores de los respectivos \textit{datasets}.
\subsection{Búsqueda multiarranque básica}

\small{\texttt{% Generator: GNU source-highlight, by Lorenzo Bettini, http://www.gnu.org/software/src-highlite
\noindent
\mbox{}\textbf{\textcolor{Blue}{def}}\ BMB\textcolor{BrickRed}{()}\textcolor{Red}{\{} \\
\mbox{}\ \ \ \ \textit{\textcolor{Brown}{\#\ n\ el\ número\ de\ predictores\ de\ cada\ dataset}} \\
\mbox{}\ \ \ \ mascaras\ \textcolor{BrickRed}{=}\ random$\_$init\textcolor{BrickRed}{(}n\textcolor{BrickRed}{)} \\
\mbox{}\ \ \ \ mascaras\ \textcolor{BrickRed}{=}\ \textcolor{BrickRed}{[}\ BL\textcolor{BrickRed}{(}m\textcolor{BrickRed}{):}\ m\ \textbf{\textcolor{Blue}{in}}\ mascaras\ \textcolor{BrickRed}{]} \\
\mbox{}\ \ \ \ tasas\ \textcolor{BrickRed}{=}\ \textcolor{BrickRed}{[}\ tasa\textcolor{BrickRed}{.}clas\textcolor{BrickRed}{(}m\textcolor{BrickRed}{):}\ m\ \textbf{\textcolor{Blue}{in}}\ mascaras\ \textcolor{BrickRed}{]} \\
\mbox{} \\
\mbox{}\ \ \ \ \textbf{\textcolor{Blue}{return}}\ mascaras\ \textcolor{BrickRed}{[}arg\ max\ tasas\textcolor{BrickRed}{]} \\
\mbox{}\textcolor{Red}{\}} \\
\mbox{}
}}
\normalsize

\subsection{GRASP}

\small{\texttt{% Generator: GNU source-highlight, by Lorenzo Bettini, http://www.gnu.org/software/src-highlite
\noindent
\mbox{}\textbf{\textcolor{Blue}{def}}\ GRASP\textcolor{BrickRed}{()}\textcolor{Red}{\{} \\
\mbox{}\ \ \ \ mascaras\ \textcolor{BrickRed}{=}\ \textcolor{BrickRed}{[}\ random$\_$greedy\textcolor{BrickRed}{(}n\textcolor{BrickRed}{)}\ from\ \textcolor{Purple}{1}\ to\ max$\_$arranques\ \textcolor{BrickRed}{]} \\
\mbox{}\ \ \ \ mascaras\ \textcolor{BrickRed}{=}\ \textcolor{BrickRed}{[}\ BL\textcolor{BrickRed}{(}m\textcolor{BrickRed}{):}\ m \textbf{\textcolor{Blue}{$\in$}}\ mascaras\ \textcolor{BrickRed}{]} \\
\mbox{}\ \ \ \ tasas\ \textcolor{BrickRed}{=}\ \textcolor{BrickRed}{[}\ tasa\textcolor{BrickRed}{.}clas\textcolor{BrickRed}{(}m\textcolor{BrickRed}{):}\ m \textbf{\textcolor{Blue}{$\in$}}\ mascaras\ \textcolor{BrickRed}{]} \\
\mbox{} \\
\mbox{}\ \ \ \ \textbf{\textcolor{Blue}{return}}\ mascaras\ \textcolor{BrickRed}{[}arg\ max\ tasas\textcolor{BrickRed}{]} \\
\mbox{}\textcolor{Red}{\}} \\
\mbox{}
}}
\normalsize

\subsection{Búsqueda Local Reiterada (ILS)}

\small{\texttt{% Generator: GNU source-highlight, by Lorenzo Bettini, http://www.gnu.org/software/src-highlite
\noindent
\mbox{}\textbf{\textcolor{Blue}{def}}\ ILS\textcolor{BrickRed}{()}\textcolor{Red}{\{} \\
\mbox{}\ \ \ \ mejor$\_$mascara\ \textcolor{BrickRed}{=}\ \textcolor{BrickRed}{[}\ random\textcolor{BrickRed}{(}\textcolor{Red}{\{}\textcolor{Purple}{1}\textcolor{BrickRed}{,}\textcolor{Purple}{0}\textcolor{Red}{\}}\textcolor{BrickRed}{)}\ from\ \textcolor{Purple}{0}\ to\ n\ \textcolor{BrickRed}{]} \\
\mbox{}\ \ \ \ num$\_$mutaciones\ \textcolor{BrickRed}{=}\ prob$\_$mutacion\ \textcolor{BrickRed}{*}\ n \\
\mbox{} \\
\mbox{}\ \ \ \ \textbf{\textcolor{Blue}{while}}\textcolor{BrickRed}{(}n$\_$eval\ \textcolor{BrickRed}{\textless{}}\ max\textcolor{BrickRed}{.}arranques\textcolor{BrickRed}{)}\textcolor{Red}{\{} \\
\mbox{}\ \ \ \ \ \ \ \ mascara\ \textcolor{BrickRed}{\textless{}-}\ BL\textcolor{BrickRed}{(}mascara\textcolor{BrickRed}{)} \\
\mbox{} \\
\mbox{}\ \ \ \ \ \ \ \ \textbf{\textcolor{Blue}{if}}\ \textcolor{BrickRed}{(}tasa\textcolor{BrickRed}{(}mascara\textcolor{BrickRed}{)}\ \textcolor{BrickRed}{\textgreater{}}\ tasa\textcolor{BrickRed}{(}mejor$\_$mascara\textcolor{BrickRed}{))}\textcolor{Red}{\{} \\
\mbox{}\ \ \ \ \ \ \ \ \ \ \ \ mejor$\_$mascara\ \textcolor{BrickRed}{=}\ mascara \\
\mbox{}\ \ \ \ \ \ \ \ \textcolor{Red}{\}} \\
\mbox{} \\
\mbox{}\ \ \ \ \ \ \ \ mutaciones\ \textcolor{BrickRed}{=}\ \textcolor{BrickRed}{[}random\textcolor{BrickRed}{(}\textcolor{Red}{\{}\textcolor{Purple}{1}\textcolor{BrickRed}{...}n\textcolor{Red}{\}}\textcolor{BrickRed}{)}\ from\ \textcolor{Purple}{0}\ to\ num$\_$mutaciones\ \textcolor{BrickRed}{]} \\
\mbox{} \\
\mbox{}\ \ \ \ \ \ \ \ foreach\ m\ \textbf{\textcolor{Blue}{in}}\ mutaciones\textcolor{Red}{\{} \\
\mbox{}\ \ \ \ \ \ \ \ \ \ \ \ mascara\ \textcolor{BrickRed}{=}\ flip\textcolor{BrickRed}{(}mascara\textcolor{BrickRed}{,}\ m\textcolor{BrickRed}{)} \\
\mbox{}\ \ \ \ \ \ \ \ \textcolor{Red}{\}} \\
\mbox{}\ \ \ \ \textcolor{Red}{\}} \\
\mbox{}\ \ \ \ \textbf{\textcolor{Blue}{return}}\ mascara \\
\mbox{}\textcolor{Red}{\}} \\
\mbox{}
}}
\normalsize

\section{Algoritmo de comparación: SFS}
\small{\texttt{% Generator: GNU source-highlight, by Lorenzo Bettini, http://www.gnu.org/software/src-highlite
\noindent
\mbox{}\textbf{\textcolor{Blue}{def}}\ SFS\textcolor{BrickRed}{()}\textcolor{Red}{\{} \\
\mbox{}\ \ \ \ non$\_$selected\ \textcolor{BrickRed}{=}\ \textcolor{Red}{\{}\textcolor{Purple}{1}\textcolor{BrickRed}{,...}n\textcolor{Red}{\}} \\
\mbox{}\ \ \ \ mascara\ \textcolor{BrickRed}{=}\ \textcolor{Red}{\{}\textcolor{Purple}{0}\textcolor{BrickRed}{,}\textcolor{Purple}{0}\textcolor{BrickRed}{...}\textcolor{Purple}{0}\textcolor{Red}{\}} \\
\mbox{}\ \ \ \ mejora\ \textcolor{BrickRed}{=}\ \textbf{\textcolor{Blue}{true}} \\
\mbox{} \\
\mbox{}\ \ \ \ \textbf{\textcolor{Blue}{do}}\textcolor{Red}{\{} \\
\mbox{}\ \ \ \ \ \ \ \ j\ \textcolor{BrickRed}{=}\ max\ arg\textcolor{Red}{\{}tasa\textcolor{BrickRed}{(}flip\textcolor{BrickRed}{(}mascara\textcolor{BrickRed}{,}x\textcolor{BrickRed}{)):}\ x\$\textcolor{BrickRed}{\textbackslash{}}\textbf{\textcolor{Blue}{in}}\$\ non$\_$selected\textcolor{Red}{\}} \\
\mbox{} \\
\mbox{}\ \ \ \ \ \ \ \ \textbf{\textcolor{Blue}{if}}\ \textcolor{BrickRed}{(}tasa\textcolor{BrickRed}{(}flip\textcolor{BrickRed}{(}mascara\textcolor{BrickRed}{,}j\textcolor{BrickRed}{))}\ \textcolor{BrickRed}{\textgreater{}}\ tasa\textcolor{BrickRed}{(}mascara\textcolor{BrickRed}{))}\textcolor{Red}{\{} \\
\mbox{}\ \ \ \ \ \ \ \ \ \ \ \ non$\_$selected\ \textcolor{BrickRed}{=}\ non$\_$selected\ \textcolor{BrickRed}{-}\ \textcolor{Red}{\{}j\textcolor{Red}{\}} \\
\mbox{}\ \ \ \ \ \ \ \ \ \ \ \ mascara\ \textcolor{BrickRed}{[}j\textcolor{BrickRed}{]}\ \textcolor{BrickRed}{=}\ \textcolor{Purple}{1} \\
\mbox{}\ \ \ \ \ \ \ \ \textcolor{Red}{\}} \\
\mbox{}\ \ \ \ \ \ \ \ \textbf{\textcolor{Blue}{else}} \\
\mbox{}\ \ \ \ \ \ \ \ \ \ \ \ mejora\ \textcolor{BrickRed}{=}\ \textbf{\textcolor{Blue}{false}} \\
\mbox{}\ \ \ \ \textcolor{Red}{\}}\textbf{\textcolor{Blue}{while}}\ mejora \\
\mbox{} \\
\mbox{}\ \ \textbf{\textcolor{Blue}{return}}\ mascara \\
\mbox{}\textcolor{Red}{\}} \\
\mbox{}
}}

Partiendo de una selección de características vacía (máscara nula), va añadiendo a cada paso la característica
de las no escogidas hasta el momento que maximiza la tasa de clasificación añadiéndola a la solución, hasta
que la característica escogida no mejore a la mejor solución hasta el momento

\section{Implementación de la práctica}
Para la implementación de la práctica, se ha optado, al igual que en la primera práctica de trayectorias simples,
por usar el lenguaje de programación R, reutilizando toda la lectura de ficheros ya implementada para la primera práctica

El clasificador \texttt{3nn} leaving one out empleado es el incluido en el paquete \texttt{class}, de nombre
\texttt{knn.cv}. Para su uso, se le pasa como argumento el número de vecinos a considerar (3). El argumento 
\texttt{use.all=TRUE} indica que en caso de empate considere todas las etiquetas de las instancias con distancias 
iguales a la mayoritaria (en este caso las tres), y escoja la etiqueta mayoritaria de entre todas. 

Ya se comentó en la primera práctica la desventaja de usar este paquete de R en caso de triple empate al escoger la etiqueta
que asignar de manera aleatoria.

Se emplea de nuevo la función (\texttt{normalize(data.frame)}) que limpia los \textit{datasets} de columnas con
un único valor, establece los nombres de los atributos todos a minúsculas, reordena los atributos del dataset para que
la clase sea el último de todos, y hace una transformación con las columnas para que todos los valores estén comprendidos
entre 0 y 1.

Se describe a continuación un esquema de los ficheros de código:
\begin{itemize}
 \item \hrefr{main.r}: de arriba a abajo, carga los paquetes necesarios, incluye el código fuente de otros ficheros
  (\hrefr{aux.r}, \hrefr{NN3.r}, \hrefr{SFS.r}, \hrefr{BL.r},\ldots) y lee los parámetros necesarios, entre ellos las 
  semillas aleatorias (12345678, 23456781, 34567812, 45678123, 56781234) del fichero \hrefr{params.r}
  
  Se ejecuta la función \texttt{cross.eval(algoritmo)}, que almacena los resultados de la ejecución en la lista
  \texttt{algoritmo.results}. Esta lista contendrá 3 dataframes con los datos del 5x2 cross validation, uno por dataset, 
  con 10 filas cada uno  y columnas las tasas de clasificación de test y train, la tasa de reducción y el tiempo de 
  ejecución. Incluye también 3 datasets más con la media de los datos anteriores para cada conjunto de datos.
 
 \item \hrefr{aux.r}: contiene las implementaciones de la función de normalización de datasets, la función de
 particionado de \texttt{data.frames}, la función objetivo \texttt{tasa.clas} y la función \texttt{cross.eval}
 descrita arriba.
 
 \item \hrefr{BL.r}, \hrefr{BMB.r}, \hrefr{GRASP.r}, \hrefr{ILS.r}: contiene la implementación de los algoritmos de Búsqueda Local (BL), Búsqueda 
 Multiarranque Básica(BMB), Búsqueda Multiarranque GRASP(GRASP) y Búsqueda Local Reiterada(ILS), respectivamente. 
 Asimismo también incluye los inicializadores de máscaras aleatorias (\texttt{random.init}), en \texttt{BL.r} 
 y de máscaras greedy aleatorizadas (\texttt{random.greedy.init}), este en \texttt{GRASP.r}
 
 \item \hrefr{params.r}: fichero del que se leen los parámetros:
  \begin{itemize}
    \item \texttt{semilla}: vector de semillas aleatorias para poder reproducir los análisis hechos.
    \item \texttt{BMB.num.sols.init, GRASP.num.sols.init, ILS.num.sols.init}: número de rearranques del algoritmo homónimo.
    Por defecto seteados a 25.
    \item \texttt{GRASP.alpha}: tolerancia para aceptar, en el algoritmo GRASP, una solución en la lista de candidatos 
    respecto a la mejor solución en esa iteración. Por defecto, $\alpha=0.3$
    \item \texttt{ILS.coef.mutacion}: multiplicado por $n$ (número de características en total), es el número de coeficientes
    de una máscara que se mutarán en la Búsqueda Local Reiterada. En nuestro caso, 0.1 por defecto (un 10\%).
  \end{itemize}
 \end{itemize}
 
 Para verificar los datos aportados para el algoritmo \texttt{algx}, a saber, basta ejecutar todo el fichero \texttt{main.r}
 hasta justo antes de la sección \textit{Obtención de resultados}. A continuación, si se ejecuta la línea 
 \texttt{algx.results <- cross.eval(algx)} se obtienen almacenados en una lista los valores de tasas y tiempos de ejecución
 aportados en la presente memoria. 
 
 Para trabajar correctamente, el working path debe estar seteado a la carpeta de códigos fuente. Se puede consultar el 
 valor actual mediante \texttt{getwd()} y setearlo mediante \texttt{setwd(path)}
 
 Cuando se cambia algo en algún algoritmo o en \texttt{params.r} hay que recargar en \texttt{main.r}
 la línea del \texttt{source(file=...)} correspondiente.
 
 \section{Experimentos y análisis de resultados}
 \subsection{Tablas de resultados}
 \begin{table}[H]	
    \caption*{Resultados del 3NN}
    \begin{adjustbox}{width=1.1\textwidth}
    \begin{tabular}{|c|r|r|r|r|r|r|r|r|r|r|r|r|}
    \hline
    \multicolumn{1}{|l|}{} & \multicolumn{ 4}{c|}{\textbf{\textit{Wdbc}}} & \multicolumn{ 4}{c|}{\textbf{\textit{Movement\_Libras}}} & \multicolumn{ 4}{c|}{\textbf{\textit{Arrhythmia}}} \\ \hline
    & \multicolumn{1}{c|}{\textbf{\textit{\%\_clas test}}} & \multicolumn{1}{c|}{\textbf{\textit{\%\_clas train}}} & \multicolumn{1}{c|}{\textbf{\textit{tasa\_red}}} & \multicolumn{1}{c|}{\textbf{T(s)}} & \multicolumn{1}{c|}{\textbf{\textit{\%\_clas test}}} & \multicolumn{1}{c|}{\textbf{\textit{\%\_clas train}}} & \multicolumn{1}{c|}{\textbf{\textit{tasa\_red}}} & \multicolumn{1}{c|}{\textbf{T(s)}} & \multicolumn{1}{c|}{\textbf{\textit{\%\_clas test}}} & \multicolumn{1}{c|}{\textbf{\textit{\%\_clas train}}} & \multicolumn{1}{c|}{\textbf{\textit{tasa\_red}}} & \multicolumn{1}{c|}{\textbf{T(s)}} \\ \hline
    \textbf{Partición 1-1} & 95.78947 & 97.53521 & 0.00000 & 0.00000 & 65.00000 & 66.66667 & 0.00000 & 0.00000 & 65.46392 & 65.62500 & 0.00000 & 0.00000 \\ \hline
    \textbf{Partición 1-2} & 97.53521 & 95.78947 & 0.00000 & 0.00000 & 66.11111 & 67.77778 & 0.00000 & 0.00000 & 65.62500 & 65.97938 & 0.00000 & 0.00000 \\ \hline
    \textbf{Partición 2-1} & 97.19298 & 95.77465 & 0.00000 & 0.00000 & 72.22222 & 64.44444 & 0.00000 & 0.00000 & 62.88660 & 61.45833 & 0.00000 & 0.00000 \\ \hline
    \textbf{Partición 2-2} & 95.77465 & 97.19298 & 0.00000 & 0.00000 & 63.33333 & 70.00000 & 0.00000 & 0.00000 & 63.02083 & 63.91753 & 0.00000 & 0.00000 \\ \hline
    \textbf{Partición 3-1} & 96.14035 & 96.47887 & 0.00000 & 0.00000 & 72.77778 & 65.00000 & 0.00000 & 0.00000 & 62.37113 & 64.06250 & 0.00000 & 0.00000 \\ \hline
    \textbf{Partición 3-2} & 96.47887 & 96.14035 & 0.00000 & 0.00000 & 63.33333 & 75.00000 & 0.00000 & 0.00000 & 63.54167 & 62.88660 & 0.00000 & 0.00000 \\ \hline
    \textbf{Partición 4-1} & 95.43860 & 97.88732 & 0.00000 & 0.00000 & 74.44444 & 66.66667 & 0.00000 & 0.00000 & 64.94845 & 62.50000 & 0.00000 & 0.00000 \\ \hline
    \textbf{Partición 4-2} & 97.88732 & 95.43860 & 0.00000 & 0.00000 & 64.44444 & 72.77778 & 0.00000 & 0.00000 & 61.45833 & 62.88660 & 0.00000 & 0.00000 \\ \hline
    \textbf{Partición 5-1} & 96.49123 & 96.83099 & 0.00000 & 0.00000 & 63.33333 & 68.33333 & 0.00000 & 0.00000 & 61.85567 & 61.45833 & 0.00000 & 0.00000 \\ \hline
    \textbf{Partición 5-2} & 96.83099 & 96.49123 & 0.00000 & 0.00000 & 67.77778 & 65.55556 & 0.00000 & 0.00000 & 60.41667 & 62.37113 & 0.00000 & 0.00000 \\ \hline
    \textbf{Media} & 96.55597 & 96.55597 & 0.00000 & 0.00000 & 67.27778 & 68.22222 & 0.00000 & 0.00000 & 63.15883 & 63.31454 & 0.00000 & 0.00000 \\ \hline
    \end{tabular}
    \end{adjustbox}
    \label{NN3}
  \end{table}
  
   \begin{table}[H]	
    \caption*{Resultados del SFS}
    \begin{adjustbox}{width=1.1\textwidth}
   \begin{tabular}{|c|r|r|r|r|r|r|r|r|r|r|r|r|}
    \hline
    \multicolumn{1}{|l|}{} & \multicolumn{ 4}{c|}{\textbf{\textit{Wdbc}}} & \multicolumn{ 4}{c|}{\textbf{\textit{Movement\_Libras}}} & \multicolumn{ 4}{c|}{\textbf{\textit{Arrhytmia}}} \\ \hline
    \multicolumn{1}{|l|}{} & \multicolumn{1}{c|}{\textbf{\textit{\%\_cl\_test}}} & \multicolumn{1}{c|}{\textbf{\textit{\%\_cl train}}} & \multicolumn{1}{c|}{\textbf{\textit{\%\_red}}} & \multicolumn{1}{c|}{\textbf{T(s)}} & \multicolumn{1}{c|}{\textbf{\textit{\%\_cl\_test}}} & \multicolumn{1}{c|}{\textbf{\textit{\%\_cl\_train}}} & \multicolumn{1}{c|}{\textbf{\textit{\%\_red}}} & \multicolumn{1}{c|}{\textbf{T(s)}} & \multicolumn{1}{c|}{\textbf{\textit{\%\_cl\_test}}} & \multicolumn{1}{c|}{\textbf{\textit{\%\_cl\_train}}} & \multicolumn{1}{c|}{\textbf{\textit{\%\_red}}} & \multicolumn{1}{c|}{\textbf{T(s)}} \\ \hline
    \textbf{Partición 1-1} & 91.92982 & 97.53521 & 0.86667 & 0.15400 & 63.88889 & 70.55556 & 0.88889 & 1.01200 & 64.94845 & 75.00000 & 0.98419 & 1.84000 \\ \hline
    \textbf{Partición 1-2} & 95.77465 & 97.89474 & 0.80000 & 0.25600 & 65.00000 & 68.33333 & 0.87778 & 1.29800 & 75.00000 & 78.35052 & 0.98419 & 1.98900 \\ \hline
    \textbf{Partición 2-1} & 97.19298 & 97.88732 & 0.86667 & 0.15200 & 64.44444 & 66.66667 & 0.93333 & 0.57300 & 64.94845 & 76.56250 & 0.98419 & 1.95300 \\ \hline
    \textbf{Partición 2-2} & 94.01408 & 97.19298 & 0.80000 & 0.26600 & 62.22222 & 69.44444 & 0.87778 & 1.13400 & 73.43750 & 71.13402 & 0.98419 & 2.22600 \\ \hline
    \textbf{Partición 3-1} & 94.38596 & 95.77465 & 0.93333 & 0.09700 & 66.66667 & 66.11111 & 0.88889 & 1.02800 & 72.16495 & 80.20833 & 0.98024 & 2.34900 \\ \hline
    \textbf{Partición 3-2} & 91.54930 & 96.14035 & 0.90000 & 0.14500 & 62.22222 & 74.44444 & 0.88889 & 1.20200 & 67.70833 & 74.22680 & 0.98024 & 2.20000 \\ \hline
    \textbf{Partición 4-1} & 94.03509 & 97.88732 & 0.90000 & 0.12000 & 71.66667 & 72.22222 & 0.88889 & 1.31100 & 74.22680 & 76.04167 & 0.98024 & 2.02000 \\ \hline
    \textbf{Partición 4-2} & 92.25352 & 94.38596 & 0.90000 & 0.11700 & 65.55556 & 77.22222 & 0.90000 & 0.87400 & 67.70833 & 76.80412 & 0.98419 & 2.44800 \\ \hline
    \textbf{Partición 5-1} & 94.73684 & 95.77465 & 0.86667 & 0.15500 & 58.33333 & 75.55556 & 0.91111 & 0.76600 & 67.52577 & 75.52083 & 0.98419 & 2.30900 \\ \hline
    \textbf{Partición 5-2} & 90.49296 & 96.49123 & 0.86667 & 0.15300 & 65.00000 & 67.22222 & 0.90000 & 0.99900 & 70.83333 & 74.74227 & 0.98814 & 1.33000 \\ \hline
    \textbf{Media} & 93.63652 & 96.69644 & 0.87000 & 0.16150 & 64.50000 & 70.77778 & 0.89556 & 1.01970 & 69.85019 & 75.85911 & 0.98340 & 2.06640 \\ \hline
    \end{tabular}
    \end{adjustbox}
    \label{SFS}
  \end{table}
  \begin{table}[H]

  \caption*{Resultados globales}
  \begin{adjustbox}{width=1.1\textwidth}
  \begin{tabular}{|c|r|r|r|r|r|r|r|r|r|r|r|r|}
  \hline
  \multicolumn{1}{|l|}{} & \multicolumn{ 4}{c|}{\textbf{\textit{Wdbc}}} & \multicolumn{ 4}{c|}{\textbf{\textit{Movement\_Libras}}} & \multicolumn{ 4}{c|}{\textbf{\textit{Arrhythmia}}} \\ \hline
  & \multicolumn{1}{c|}{\textbf{\textit{\%\_clas test}}} & \multicolumn{1}{c|}{\textbf{\textit{\%\_clas train}}} & \multicolumn{1}{c|}{\textbf{\textit{tasa\_red}}} & \multicolumn{1}{c|}{\textbf{T(s)}} & \multicolumn{1}{c|}{\textbf{\textit{\%\_clas test}}} & \multicolumn{1}{c|}{\textbf{\textit{\%\_clas train}}} & \multicolumn{1}{c|}{\textbf{\textit{tasa\_red}}} & \multicolumn{1}{c|}{\textbf{T(s)}} & \multicolumn{1}{c|}{\textbf{\textit{\%\_clas test}}} & \multicolumn{1}{c|}{\textbf{\textit{\%\_clas train}}} & \multicolumn{1}{c|}{\textbf{\textit{tasa\_red}}} & \multicolumn{1}{c|}{\textbf{T(s)}} \\ \hline
  \textbf{3-NN} & 96.55597 & 96.55597 & 0.00000 & 0.00000 & 67.27778 & 68.22222 & 0.00000 & 0.00000 & 63.15883 & 63.31454 & 0.00000 & 0.00000 \\ \hline
  \textbf{SFS} & 94.09402 & 96.52100 & 0.84000 & 0.17420 & 64.88889 & 72.55556 & 0.89111 & 1.01910 & 69.23002 & 75.75816 & 0.97905 & 2.35900 \\ \hline
  \textbf{BL} & 95.64183 & 97.64505 & 0.48667 & 0.23150 & 66.66667 & 68.66667 & 0.52111 & 0.98410 & 62.90217 & 64.55756 & 0.50000 & 19.60270 \\ \hline
  \textbf{ES} & 95.50186 & 98.59402 & 0.50333 & 30.45330 & 66.88889 & 70.50000 & 0.51556 & 25.48140 & 63.26085 & 68.91591 & 0.51462 & 373.45210 \\ \hline
  \textbf{BT básica} & 96.02792 & 98.91055 & 0.52667 & 44.06440 & 67.38889 & 74.11111 & 0.53444 & 70.71810 & 63.57013 & 72.54564 & 0.53755 & 514.00300 \\ \hline
  \textbf{BT extendida} & 95.64196 & 98.84025 & 0.50667 & 53.01520 & 67.44444 & 72.05556 & 0.49333 & 99.33300 & 62.79693 & 69.58763 & 0.47391 & 3135.18590 \\ \hline
  \end{tabular}
  \end{adjustbox}
  \label{}
  \end{table}

  
  Los algoritmos deberían analizarse por dataset, ya que cada uno posee unas características muy determinadas: \texttt{wdbc} tiene muy pocos atributos(30), pero muchas observaciones(569),
  \texttt{movement libras} tiene menos observaciones(360) pero el triple de atributos(90). Y \texttt{arrhythmia} es de todos el dataset más duro de procesar (386 instancias por 253 atributos
  significativos). Aunque recordamos que para calcular la selección de características se hace con una partición al 50\% estratificada, lo que nos lleva a considerar la mitad de instancias.
  Por tanto analizaremos los resultados por dataset.
  
  \subsection{Wdbc}
  
  Es un problema de clasificación binaria.
  
  \imagen{../data/wdbc.png}{Tasas de clasificación en Wdbc}{wdbcgraph}{0.7}
  
  Observamos que el algoritmo que mejores tasas de clasificación obtiene es el 3NN. Las tasas que obtiene en test y train
  son similares, y por tanto no está efectuando overfitting. Cuando hacemos un boxplot de las variables que aporta el SFS
  ejecutado sobre ese dataset, nos encontramos que en la mayoría de esas variables los rangos intercuartílicos de ambas
  clases son disjuntos, por lo que las variables caracterizan muy bien a una clase o a otra, y por tanto cuando tomamos los
  vecinos más cercanos, podemos inferir que cuando la distancia entre las características de la instancia a clasificar y la ya 
  clasificada de entre las que nos da la máscara será muy pequeña, y por lo explicado sobre el boxplot, es muy probable que
  ambas instancias tengan la misma clase.
  
  \imgn{../data/wdbc3nn.png}{Selección de características del SFS}{1}
  
  La SFS aporta una tasa de clasificación media del 94\% en test, y aunque es la más baja de todos los algoritmos considerados,
  tiene la tasa de reducción más alta (del 84\%). Por tanto si tuviéramos muchísimas más instancias de este dataset, dependiendo
  de las necesidades y siempre que busquemos rapidez de cálculo, no sería una mala alternativa frente al resto de algoritmos.
  
  De las otras metaheurísticas la que mejor funciona es la búsqueda tabú. De hecho, podríamos observar que hace algo de overfitting
  en las tasas de clasificación del conjunto de entrenamiento, llegando a obtener 100\% de tasa de clasificación en un caso.
  Si tomamos un boxplot de todas las variables que tiene el dataset (normalizadas, claro está), podemos inferir que la búsqueda
  tabú a corto plazo funciona tan bien porque hay variables cuya intersección de rangos intercuartílicos es bastante grande,
  como se puede apreciar en la figura aportada, y otras cuyos rangos intercuartílicos son completamente disjuntos. El mecanismo
  de la tabú hace que si hacemos un flip de una variable que no mejora la tasa de clasificación (a saber, es probable que
  sea una con los rangos intercuartílicos de ambas clases poco separados), ese movimiento no se vuelva a hacer en un tiempo,
  y por tanto nos vamos a ir quedando con variables que caracterizan muy bien a las clases, y la lista tabú va a estar repleta
  de movimientos para variables que no aportan información relevante sobre la clase. Además, en caso de que el movimiento
  sí sea bueno y entre en la lista tabú, será porque caracteriza bien a las clases, y agregándole otro movimiento bueno podríamos
  tener una buena regla de asociación que caracterizase bien las instancias.
  
  \imgn{../data/wdbctabu.png}{Boxplot de variables del wdbc}{1}
  
  Podemos afirmar a raíz de los resultados que la tabú con memoria a largo plazo también funciona muy bien para el dataset,
  obteniéndose unos resultados similares a la tabú con memoria a corto plazo, aunque efectúa un poco de overfitting en la
  primera partición de todas.
  
  El enfriamiento simulado también aporta buenos resultados para el dataset. Tanto ES, como BT como BT extendida tienen un
  porcentaje de reducción del 50\%, lejos del 80\% de la SFS, y tardan tiempos no superiores al minuto en ejecutarse por partición,
  y por tanto asumibles.
  
  La búsqueda local produce unos resultados muy similares a los de la SFS excepto en tasa de reducción. Podemos pensar que
  lo que la hace no obtener resultados tan buenos como las tabú es que tiene en cuenta un paso de temporalidad (a cada momento
  se coge la característica que mejora la solución que tiene, escogiendo ésta aleatoriamente), en vez de tener en cuenta varios
  pasos de temporalidad que es lo que hace la tabú al llevar su lista tabú (esto es, la suma de varias características en conjunción
  podría caracterizar muy bien las instancias, mientras que la BL las intenta caracterizar mejor modificando una sola característica 
  a cada paso). En definitiva, la búsqueda tabú tiene más en cuenta el histórico de la exploración.
 
  \subsection{Movement libras}
  
  Se trata de un problema de clasficación multiclase (15 clases después de normalizar el dataset).
  
  \imagen{../data/mlibras.png}{Tasas de clasificación en Movement Libras}{wdbcgraph}{0.7}
  
  Dado el elevado número de clases, no podemos analizar un boxplot para ver las diferencias entre los atributos que escogen unos
  algoritmos y otros, pero parece que hay cierta estabilidad en las soluciones y que es difícil pasar del 70\% de acierto. Dicho lo
  cual, es claro que en igualdad de condiciones deberíamos tomar la solución que menos tiempo tarde en efectuar los cálculos y
  mayor tasa de reducción tenga (el SFS). Si escogiéramos un algoritmo por su tasa clasificatoria, nos iríamos a la tabú search
  extendida, aunque es bastante lenta (1 minuto y 40 por partición).
  
  Además, todos los algoritmos tienen particiones en las que efectúan bastante overfitting (de diferencias de un 10\% o más).
  Esto lo podemos explicar porque tenemos muchos más atributos que en el caso del \texttt{wdbc}, concretamente el triple, y
  casi la mitad de instancias. Por tanto hay poca densidad de datos y las tasas en entrenamiento son mejores que las de 
  prueba (7\% en las medias de la búsqueda tabú por ejemplo).
  
  \subsection{Arrhythmia}
  
  Se trata de un problema de clasficación multiclase (5 clases después de normalizar el dataset).
  
  \imagen{../data/arrhythmia.png}{Tasas de clasificación en Arrhythmia}{wdbcgraph}{0.7}
  
  Es un problema muy complejo, pues tenemos menos instancias que variables para predecir (193 instancias que tendrán
  los conjuntos de entrenamiento por 253 variables). La SFS ya nos está diciendo, obteniendo datos mejores de clasificación
  que el propio 3-knn, que la mayoría de variables no son estadísticamente significativas (reduce casi un 98\% el conjunto de variables)
  para efectuar la clasificación.
  
  En el resto de casos las tasas de reducción son mucho más bajas (en torno al 50\%), pero con porcentajes de clasificación
  mucho menores. Probablemente, en el caso de las búsquedas tabú falten iteraciones para hacer converger la solución a una mejor
  tasa de clasificación. En general se observa una gran dependencia de las 4 metaheaurísticas empleadas de la solución aleatoria
  inicial, que parece ser bastante mala (de hecho, al ser aleatoria, tendrá un número equiparable de 0s y de 1s, aunque la mayoría
  de variables no son estadísticamente significativas para determinar la clasificación, y nos deberíamos centrar en aquellas
  que sí lo son y intentar modificar la solución desde ahí. Una posible solución sería tomar como solución inicial para las metaheurísticas
  la del propio algoritmo greedy.
  
  Observamos en la tabla de resultados de la tabú extendida que hay una partición que tarda casi el doble que el resto.
  A juzgar por cómo hemos programado el algoritmo, sólo cabe la posibilidad de que tarde más porque entra más veces a la
  reinicialización, y concretamente a la diversificación, aunque el dato puede ser simplemente un outlier.
  
  Un porcentaje del 75\% de diversificar(de las dos formas posibles) en este caso parece bastante
  elevado para partir de una solución aleatoria o generada a partir de frecuencias que será bastante mala en la mayoría de los casos. Maxime cuando el problema
  tiene 253 posibles variables. Como por el operador de generación de vecino que hemos montado, y eso se observa en la tabla
  resumen, tenemos la misma probabilidad de unos que de ceros en una solución, y hay en este caso $\binom{253}{50}$ posibles
  máscaras que obtendremos, por $\binom{30}{15}$, en el que las búsquedas tabú sí eran muy buenas,
  del caso del \texttt{movement libras}, podemos afirmar que quizá falten iteraciones para poder hacer esta búsqueda tabú 
  converger a soluciones mejores. Pero si cada tabú extendida está tardando del orden de 50-60 minutos por partición,
  un número mayor de iteraciones es insostenible para un computador doméstico.
  
  \subsection{Notas finales}
  
  Una posible solución a nuestros problemas de convergencia sería programar otro operador de generación de vecino, que
  no tuviese la misma probabilidad de generar unos que de ceros. Como las tasas de reducción de la SFS en todos los casos
  es superior al 80\%, llegando al 97\% en el caso de arrhythmia, podríamos intentar programar un operador de vecino que
  tuviese un 70-75\% de posibilidades de generar un 0 en la posición escogida, y un 1 con un 0.3-0.25 de probabilidad.
  
  Otra mejora que podría implementarse es partir en todos los casos como solución inicial de la greedy.
  
  En el caso del dataset \texttt{mlibras}, modificando la tenencia tabú a $2/3*n$ (esto es\texttt{BT.coef.tenencia.tabu <- 2/3}
  en \texttt{params.r}) se ha llegado a un 68\% de tasa de acierto.
  
  
\end{document}