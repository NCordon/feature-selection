\documentclass[a4paper,11pt]{article}
% Símbolo del euro
\usepackage[gen]{eurosym}
% Codificación
\usepackage[utf8]{inputenc}
% Idioma
\usepackage[spanish]{babel} % English language/hyphenation
\selectlanguage{spanish}
% Hay que pelearse con babel-spanish para el alineamiento del punto decimal
\decimalpoint
\usepackage{dcolumn}
\newcolumntype{d}[1]{D{.}{\esperiod}{#1}}
\makeatletter
\addto\shorthandsspanish{\let\esperiod\es@period@code}
\makeatother

\usepackage[usenames,dvipsnames]{color} % Coloring code

\usepackage{csvsimple}
\usepackage{adjustbox}
\newsavebox\ltmcbox


% Para matrices
\usepackage{amsmath}

% Símbolos matemáticos
\usepackage{amssymb}
\let\oldemptyset\emptyset
\let\emptyset\varnothing

% Hipervínculos
\usepackage{url}

\usepackage[section]{placeins} % Para gráficas en su sección.
\usepackage{mathpazo} % Use the Palatino font
\usepackage[T1]{fontenc} % Required for accented characters
\newenvironment{allintypewriter}{\ttfamily}{\par}
\setlength{\parindent}{0pt}
\parskip=8pt
\linespread{1.05} % Change line spacing here, Palatino benefits from a slight increase by default


% Imágenes
\usepackage{graphicx}
\usepackage{float}
\usepackage{caption}
\usepackage{wrapfig} % Allows in-line images



% Márgenes
\usepackage{geometry}
 \geometry{
 a4paper,
 total={210mm,297mm},
 left=30mm,
 right=30mm,
 top=25mm,
 bottom=25mm,
 }


% Referencias
\usepackage{fncylab}
\labelformat{figure}{\textit{\figurename\space #1}}

\usepackage{hyperref}
\hypersetup{
  colorlinks   = true, %Colours links instead of ugly boxes
  urlcolor     = blue, %Colour for external hyperlinks
  linkcolor    = blue, %Colour of internal links
  citecolor   = red %Colour of citations
}


\makeatletter
\renewcommand{\@listI}{\itemsep=0pt} % Reduce the space between items in the itemize and enumerate environments and the bibliography
\newcommand{\imagent}[4]{
  \begin{wrapfigure}{#4}{0.7\textwidth}
    \begin{center}
    \includegraphics[width=0.7\textwidth]{#1}
    \end{center}
    \caption{#3}
    \label{#4}
  \end{wrapfigure}
}


\newcommand{\imagen}[4]{
  \begin{minipage}{\linewidth}
    \centering
    \includegraphics[width=#4\textwidth]{#1}
    \captionof{figure}{#2}
    \label{#3}
  \end{minipage} 
}

\newcommand{\imgn}[3]{
  \begin{minipage}{\linewidth}
    \centering
    \includegraphics[width=#3\textwidth]{#1}
    \captionof{figure}{#2}
  \end{minipage} 
}

% Ejemplo de parámetro: ILS.r
\newcommand{\hrefr}[1]{
\href{../bin/#1}{#1}
}

%Customize enumerate tag
\usepackage{enumitem}
%Sections don't get numbered
%\setcounter{secnumdepth}{0}
\newcommand{\blue}[1]{\textcolor{blue}{#1}}
\begin{document}
%%%%%%%%%%%%%%%%%%%%%%%%%%%%%%%%%%%%%%%%%
% University Assignment Title Page 
% LaTeX Template
% Version 1.0 (27/12/12)
%
% This template has been downloaded from:
% http://www.LaTeXTemplates.com
%
% Original author:
% WikiBooks (http://en.wikibooks.org/wiki/LaTeX/Title_Creation)
% Modified by: NCordon (https://github.com/NCordon)
%
% License:
% CC BY-NC-SA 3.0 (http://creativecommons.org/licenses/by-nc-sa/3.0/)
% 
% Instructions for using this template:
% This title page is capable of being compiled as is. This is not useful for 
% including it in another document. To do this, you have two options: 
%
% 1) Copy/paste everything between \begin{document} and \end{document} 
% starting at \begin{titlepage} and paste this into another LaTeX file where you 
% want your title page.
% OR
% 2) Remove everything outside the \begin{titlepage} and \end{titlepage} and 
% move this file to the same directory as the LaTeX file you wish to add it to. 
% Then add \input{./title_page_1.tex} to your LaTeX file where you want your
% title page.
%
%%%%%%%%%%%%%%%%%%%%%%%%%%%%%%%%%%%%%%%%%
\begin{titlepage}

\newcommand{\HRule}{\rule{\linewidth}{0.5mm}} % Defines a new command for the horizontal lines, change thickness here

\center % Center everything on the page
 
%----------------------------------------------------------------------------------------
%	HEADING SECTIONS
%----------------------------------------------------------------------------------------
\textsc{\LARGE Universidad de Granada}\\[1.5cm]
\textsc{\Large Metaheaurísticas}\\[0.5cm] 

%----------------------------------------------------------------------------------------
%	TITLE SECTION
%----------------------------------------------------------------------------------------
\bigskip
\HRule \\[0.4cm]
{ \huge \bfseries Práctica IV}\\[0.4cm] % Title of your document
{ \huge \bfseries Selección de características}\\
\HRule \\[1.5cm]
 
%----------------------------------------------------------------------------------------
%	AUTHOR SECTION
%----------------------------------------------------------------------------------------

\begin{minipage}{\textwidth}
\begin{center} \large
\emph{Sistema de colonia de hormigas}\\
\emph{Sistema de hormigas}\\
\end{center}
\end{minipage}

%----------------------------------------------------------------------------------------
%	LOGO SECTION
%----------------------------------------------------------------------------------------

\begin{center}
\includegraphics[width=8cm]{../data/ugr.jpg}
\end{center}
%----------------------------------------------------------------------------------------

\begin{minipage}{\textwidth}
\begin{center} \large
Ignacio Cordón Castillo, 25352973G\\
\url{nachocordon@correo.ugr.es}\\
\ \\
$4^{\circ}$ Doble Grado Matemáticas Informática\\
Grupo Prácticas Viernes
\end{center}
\end{minipage}


\vspace{\fill}% Fill the rest of the page with whitespace
\large\today
\end{titlepage}  

\newpage
\tableofcontents
\newpage
% Examples of inclussion of images
%\imagent{ugr.jpg}{Logo de prueba}{ugr}
%\imagen{ugr.jpg}{Logo de prueba}{ugr2}{size relative to the \textwidth}

\section{Descripción del problema}
Dado un \textit{dataset} de instancias ya clasificadas, con una serie de atributos, se pretende comparar las distintas 
metaheaurísticas disponibles para comprobar cuál produce el conjunto de atributos que sirven para obtener una mayor 
tasa de clasificación (número de instancias bien clasificadas sobre el total) usando un clasificador de instancias.
En otras palabras, si tenemos $p$ atributos o características, el número total de subconjuntos de características que podemos
escoger es $2^p$. Cada uno de esos conjuntos o selecciones de características tendrá una tasa de clasificación asociada para
el clasificador que hemos escogido. Como este problema cuando $p$ es muy grande es inabordable por fuerza bruta, se trata de aplicar metaheurísticas para encontrar la mejor selección de características que
podamos de entre esas $2^p$ posibilidades.

Se considera la tasa de clasificación en el problema usando un clasificador \textit{3-knn} leaving one-out. Para cada 
instancia de un conjunto, toma las 3 más cercanas usando la distancia euclídea entre sus atributos, y etiqueta la nueva
instancia en función de la etiqueta mayoritaria de entre esas tres. Se efectúan 5 particiones al 50\% estratificadas
por clase en \textit{train - test}, de modo que la metaheurística proporcionará un conjunto de atributos para el conjunto de
entrenamiento, y se calculará la tasa de clasificación que produce sobre el conjunto de prueba para el \textit{3-knn} dicho
conjunto de atributos. El proceso se repite intercambiando los conjuntos de entrenamiento y de test. La calidad de la 
metaheaurística/algoritmo se calculará como la media de todas las tasas de clasificación obtenidas sobre el conjunto test
(10 en total). Otras medidas que se tendrán en cuenta a la hora de evaluar la bondad de un algoritmo empleado serán:
\begin{itemize}
 \item Tasa de reducción: Porcentaje de características que ha eliminado la máscara sobre el total que podía 
 seleccionar. Mejor cuanto más alta. En nuestros resultados lo hemos expresado como un tanto por uno.
 \item Tiempo de ejecución: Tiempo en segundos que tarda el algoritmo en devolver un conjunto de características.
 Se busca el menor tiempo de ejecución posible.
\end{itemize}


La tasa de clasificación del clasificador se mide (en tanto por uno) como: $$tasa\: clasificacion = \frac{instancias\: 
bien\: clasificadas}{total\: instancias}$$

La representación usada es la binaria (1 o 0 por característica), donde tenemos un vector de tamaño $n$, con $n$ el número 
de atributos (exceptuando la clase), y la metaheaurística/algoritmo ha seleccionado el atributo si y solo si lo ha marcado 
a 1. A una representación de esta forma, la denominamos máscara: $$ mask =\begin{matrix} (0 & 1\ldots 1 & 1 & 0\ldots) 
\end{matrix}$$

\newpage
\section{Descripción elementos comunes de los algoritmos}
Los algoritmos considerados tienen las siguientes componentes:

\begin{itemize} 
\item \textbf{Esquema de representación:} se emplean máscaras de bits, sucesiones de unos y ceros de tanta longitud como atributos haya,
exceptuando la clase, donde un 1 en la posición $i-$ésima indica que esa característica se ha escogido, y un $0$ que no.

\item \textbf{Función objetivo}: la función a maximizar será la tasa de clasificación explicada arriba usando el clasificador
3-knn con la selección de características indicada por la máscara.\\

  \small\texttt{% Generator: GNU source-highlight, by Lorenzo Bettini, http://www.gnu.org/software/src-highlite
\noindent
\mbox{}\textbf{\textcolor{Blue}{def}}\ tasa$\_$clasificacion\ \textcolor{BrickRed}{(}train\textcolor{BrickRed}{,}\ mascara\textcolor{BrickRed}{)}\textcolor{Red}{\{} \\
\mbox{}\ \ train\ \textcolor{BrickRed}{=}\ apply\textcolor{BrickRed}{(}mascara\textcolor{BrickRed}{,}\ train\textcolor{BrickRed}{)} \\
\mbox{}\ \ clases\ \textcolor{BrickRed}{=}\ train\textcolor{BrickRed}{[}\textbf{\textcolor{Blue}{class}}\textcolor{BrickRed}{]} \\
\mbox{}\ \ train\ \textcolor{BrickRed}{=}\ train\textcolor{BrickRed}{[-}\textbf{\textcolor{Blue}{class}}\textcolor{BrickRed}{]} \\
\mbox{} \\
\mbox{}\ \ fit\ \textcolor{BrickRed}{=}\ leave\textcolor{BrickRed}{.}one\textcolor{BrickRed}{.}out\textcolor{BrickRed}{.}3knn\textcolor{BrickRed}{(}train\textcolor{BrickRed}{,}\ clases\textcolor{BrickRed}{)} \\
\mbox{} \\
\mbox{}\ \ \textbf{\textcolor{Blue}{return}}\ \textcolor{BrickRed}{(}\textcolor{Purple}{100}\ \textcolor{BrickRed}{*}\ length\textcolor{BrickRed}{(}which\textcolor{BrickRed}{(}clases\ \textcolor{BrickRed}{==}\ fit\textcolor{BrickRed}{))}\ \textcolor{BrickRed}{/}\ length\textcolor{BrickRed}{(}clases\textcolor{BrickRed}{))} \\
\mbox{}\textcolor{Red}{\}} \\
\mbox{}
}
  \normalsize
  
\item \textbf{Rastros de feromona y cálculo de la información heurísticas}

Se han escogido dos vectores para representar los rastros de feromona:

\begin{itemize}
  \item \texttt{trail.features = [$\tau_1$, \ldots $\tau_n$]}
  
    donde $n$ es el número de características del problema.
    
    Aquí se almacena la feromona correspondiente a escoger la característica $i-$ésima
    
    En \textit{Sistema de Colonia de Hormigas con Búsqueda Local (SCH-BL)} se inicializa: \texttt{trail.features = [$10^{-6},\ldots 10^{-6}$]}
    
    En \textit{Sistema de Hormigas Max-Min con Búsqueda Local(SHMM-BL)} se inicializa: \texttt{trail.features = [$\tau_{max},\ldots \tau_{max}$]},
    donde \texttt{$\tau_{max} = \frac{tasa\_objetivo(mask\_inicial)}{global\_evap}$} donde $global\_evap = 0.2$.
  \item \texttt{trail.num.features = [$\tau_{1 car}$, \ldots $\tau_{n car}$]}
  
    donde la componente $i-$ésima almacena la feromona correspondiente a escoger un número $i$ de características
    
    En ambos casos se inicializará \texttt{trail.num.features = \big[$\frac{1}{n}$, \ldots $\frac{1}{n}$\big]}
    
    donde $n$ es el número de características del problema.
  El cálculo de la entropía se efectúa de acuerdo al siguiente procedimiento:\\

    \small\texttt{% Generator: GNU source-highlight, by Lorenzo Bettini, http://www.gnu.org/software/src-highlite
\noindent
\mbox{}\textbf{\textcolor{Blue}{def}}\ heuristic\textcolor{BrickRed}{.}info\textcolor{BrickRed}{()}\textcolor{Red}{\{} \\
\mbox{}\ \ \ \ heuristic\ \textcolor{BrickRed}{=}\ \textcolor{BrickRed}{[}\textcolor{Purple}{0}\textcolor{BrickRed}{,}\textcolor{Purple}{0}\textcolor{BrickRed}{...}\textcolor{Purple}{0}\textcolor{BrickRed}{]} \\
\mbox{} \\
\mbox{}\ \ \ \ \textbf{\textcolor{Blue}{for}}\ i\ \textbf{\textcolor{Blue}{in}}\ \textcolor{Red}{\{}\textcolor{Purple}{1}\textcolor{BrickRed}{...}n\textcolor{Red}{\}\{} \\
\mbox{} \\
\mbox{}\ \ \ \ \ \ \ \ \textbf{\textcolor{Blue}{if}}\textcolor{BrickRed}{(}\ \textbf{\textcolor{Blue}{not}}\textcolor{BrickRed}{.}numeric\textcolor{BrickRed}{(}data\textcolor{BrickRed}{[,}i\textcolor{BrickRed}{])}\ \textcolor{BrickRed}{)}\textcolor{Red}{\{} \\
\mbox{}\ \ \ \ \ \ \ \ \ \ \ \ discretized$\_$values\ \textcolor{BrickRed}{=}\ split\textcolor{BrickRed}{(}data\textcolor{BrickRed}{[,}i\textcolor{BrickRed}{],}\ data\textcolor{BrickRed}{.}\textbf{\textcolor{Blue}{class}}\textcolor{BrickRed}{)} \\
\mbox{}\ \ \ \ \ \ \ \ \textcolor{Red}{\}} \\
\mbox{}\ \ \ \ \ \ \ \ \textbf{\textcolor{Blue}{else}}\textcolor{Red}{\{} \\
\mbox{}\ \ \ \ \ \ \ \ \ \ \ \ columna\ \textcolor{BrickRed}{=}\ sort\textcolor{BrickRed}{(}data\textcolor{BrickRed}{[,}i\textcolor{BrickRed}{])} \\
\mbox{}\ \ \ \ \ \ \ \ \ \ \ \ chunk\ \textcolor{BrickRed}{=}\ ceil\textcolor{BrickRed}{(}\ length\textcolor{BrickRed}{(}columna\textcolor{BrickRed}{)/}\textcolor{Purple}{10}\ \textcolor{BrickRed}{)} \\
\mbox{}\ \ \ \ \ \ \ \ \ \ \ \ discretized$\_$values\ \textcolor{BrickRed}{=}\ split\textcolor{BrickRed}{(}columna\textcolor{BrickRed}{,}\ size$\_$split\ \textcolor{BrickRed}{=}\ chunk\ \textcolor{BrickRed}{)} \\
\mbox{}\ \ \ \ \ \ \ \ \textcolor{Red}{\}} \\
\mbox{} \\
\mbox{}\ \ \ \ \ \ \ \ Nc\ \textcolor{BrickRed}{\textless{}-}\ length\textcolor{BrickRed}{(}unique\textcolor{BrickRed}{(}data\textcolor{BrickRed}{.}\textbf{\textcolor{Blue}{class}}\textcolor{BrickRed}{))} \\
\mbox{}\ \ \ \ \ \ \ \ value\ \textcolor{BrickRed}{=}\ \textcolor{Purple}{0} \\
\mbox{} \\
\mbox{}\ \ \ \ \ \ \ \ \textbf{\textcolor{Blue}{for}}\ c\ \textbf{\textcolor{Blue}{in}}\ \textcolor{Red}{\{}\textcolor{Purple}{1}\textcolor{BrickRed}{...}Nc\textcolor{Red}{\}\{} \\
\mbox{} \\
\mbox{}\ \ \ \ \ \ \ \ \ \ \ \ \textbf{\textcolor{Blue}{for}}\ f\ \textbf{\textcolor{Blue}{in}}\ discretized$\_$values\textcolor{Red}{\{} \\
\mbox{}\ \ \ \ \ \ \ \ \ \ \ \ \ \ \ \ dist$\_$f\ \textcolor{BrickRed}{=}\ data\textcolor{BrickRed}{[}data\textcolor{BrickRed}{[,}i\ \textcolor{BrickRed}{\%}\textbf{\textcolor{Blue}{in}}\textcolor{BrickRed}{\%}\ f\textcolor{BrickRed}{,}\ \textcolor{BrickRed}{]} \\
\mbox{} \\
\mbox{}\ \ \ \ \ \ \ \ \ \ \ \ \ \ \ \ prob$\_$c$\_$f\ \textcolor{BrickRed}{\textless{}-}\ sum\textcolor{BrickRed}{(}which\textcolor{BrickRed}{(}dist$\_$f\textcolor{BrickRed}{.}\textbf{\textcolor{Blue}{class}}\ \textcolor{BrickRed}{==}\ c\textcolor{BrickRed}{))}\ \textcolor{BrickRed}{/}\ length\textcolor{BrickRed}{(}f\textcolor{BrickRed}{)} \\
\mbox{}\ \ \ \ \ \ \ \ \ \ \ \ \ \ \ \ prob$\_$c\ \textcolor{BrickRed}{\textless{}-}\ sum\textcolor{BrickRed}{(}which\textcolor{BrickRed}{(}data\textcolor{BrickRed}{.}\textbf{\textcolor{Blue}{class}}\ \textcolor{BrickRed}{==}\ c\textcolor{BrickRed}{))}\ \textcolor{BrickRed}{/}\ length\textcolor{BrickRed}{(}columna\textcolor{BrickRed}{)} \\
\mbox{}\ \ \ \ \ \ \ \ \ \ \ \ \ \ \ \ prob$\_$f\ \textcolor{BrickRed}{\textless{}-}\ length\textcolor{BrickRed}{(}dist$\_$f\textcolor{BrickRed}{)}\ \textcolor{BrickRed}{/}\ length\textcolor{BrickRed}{(}columna\textcolor{BrickRed}{)} \\
\mbox{} \\
\mbox{}\ \ \ \ \ \ \ \ \ \ \ \ \ \ \ \ \textbf{\textcolor{Blue}{if}}\ \textcolor{BrickRed}{(}prob$\_$c$\_$f\ \textcolor{BrickRed}{\textgreater{}}\ \textcolor{Purple}{0}\textcolor{BrickRed}{)}\textcolor{Red}{\{} \\
\mbox{}\ \ \ \ \ \ \ \ \ \ \ \ \ \ \ \ \ \ \ \ value\ \textcolor{BrickRed}{+=}\ prob\textcolor{BrickRed}{.}c\textcolor{BrickRed}{.}f\ \textcolor{BrickRed}{*}\ log$\_$2\textcolor{BrickRed}{(}prob\textcolor{BrickRed}{.}c\textcolor{BrickRed}{.}f\ \textcolor{BrickRed}{/}\ \textcolor{BrickRed}{(}prob\textcolor{BrickRed}{.}c\ \textcolor{BrickRed}{*}\ prob\textcolor{BrickRed}{.}f\textcolor{BrickRed}{))} \\
\mbox{}\ \ \ \ \ \ \ \ \ \ \ \ \ \ \ \ \textcolor{Red}{\}} \\
\mbox{}\ \ \ \ \ \ \ \ \ \ \ \ \textcolor{Red}{\}} \\
\mbox{}\ \ \ \ \ \ \ \ \textcolor{Red}{\}} \\
\mbox{}\ \ \ \ \ \ \ \ heuristic\textcolor{BrickRed}{[}i\textcolor{BrickRed}{]}\ \textcolor{BrickRed}{=}\ value \\
\mbox{}\ \ \ \ \textcolor{Red}{\}} \\
\mbox{}\ \ \ \ \textbf{\textcolor{Blue}{return}}\ heuristic \\
\mbox{}\textcolor{Red}{\}} \\
\mbox{}
}
    \normalsize  
  
  \item \textbf{Proceso de generación de soluciones}:
  
    \small\texttt{% Generator: GNU source-highlight, by Lorenzo Bettini, http://www.gnu.org/software/src-highlite
\noindent
\mbox{}\textbf{\textcolor{Blue}{def}}\ init$\_$iteration\textcolor{BrickRed}{(}data\textcolor{BrickRed}{)}\textcolor{Red}{\{} \\
\mbox{}\ \ \ \ \textit{\textcolor{Brown}{\#\ cada\ path\ tiene\ longitud\ n,\ número\ de\ características}} \\
\mbox{}\ \ \ \ paths\ \textcolor{BrickRed}{=}\ \textcolor{BrickRed}{[}\ permutation\textcolor{BrickRed}{[}\textcolor{Purple}{1}\textcolor{BrickRed}{,}\textcolor{Purple}{0}\textcolor{BrickRed}{,}\textcolor{Purple}{0}\textcolor{BrickRed}{,}\textcolor{Purple}{0}\textcolor{BrickRed}{...,}\textcolor{Purple}{0}\textcolor{BrickRed}{]}\ \textbf{\textcolor{Blue}{for}}\ i\ \textbf{\textcolor{Blue}{in}}\ \textcolor{Red}{\{}\textcolor{Purple}{1}\textcolor{BrickRed}{...}num$\_$ants\textcolor{Red}{\}}\ \textcolor{BrickRed}{]} \\
\mbox{} \\
\mbox{}\ \ \ \ \textit{\textcolor{Brown}{\#\ Numero\ de\ características\ que\ escogerá\ cada\ hormiga}} \\
\mbox{}\ \ \ \ num$\_$features\ \textcolor{BrickRed}{=}\ \ \textcolor{BrickRed}{[}\ sample\textcolor{BrickRed}{(}\ \textcolor{Purple}{0}\textcolor{BrickRed}{:}n\textcolor{BrickRed}{-}\textcolor{Purple}{1}\textcolor{BrickRed}{,}\ prob\textcolor{BrickRed}{=}trail$\_$num$\_$features\textcolor{BrickRed}{)} \\
\mbox{}\ \ \ \ \ \ \ \ \ \ \ \ \ \ \ \ \ \ \ \ \ \ \ \ \ \ \ \ \ \textbf{\textcolor{Blue}{for}}\ i\ \textbf{\textcolor{Blue}{in}}\ \textcolor{Red}{\{}\textcolor{Purple}{1}\textcolor{BrickRed}{...}num$\_$ants\textcolor{Red}{\}}\ \textcolor{BrickRed}{]} \\
\mbox{} \\
\mbox{}\textcolor{Red}{\}} \\
\mbox{}
}
    \normalsize  
  
  Situamos cada hormiga en una característica inicial, y seleccionamos al azar el número de características que escogera
  cada hormiga, usando como ruleta \texttt{trail\_num\_features}

  \item \textbf{Algoritmo Búsqueda Local (BL)} empleado:
  
  El operador de generación de vecino usado ha sido:
  \footnote{En R, puesto que estamos manejando un vector de 0 y 1, el operador \texttt{flip} puede explicitarse como
  \texttt{mask[j]} = \texttt{!mask[j]}}
  
  \small\texttt{% Generator: GNU source-highlight, by Lorenzo Bettini, http://www.gnu.org/software/src-highlite
\noindent
\mbox{}\textbf{\textcolor{Blue}{def}}\ flip\textcolor{BrickRed}{(}mask\textcolor{BrickRed}{,}j\textcolor{BrickRed}{)}\textcolor{Red}{\{} \\
\mbox{}\ \ \ \ mask\textcolor{BrickRed}{[}j\textcolor{BrickRed}{]}\ \textcolor{BrickRed}{=}\ \textcolor{BrickRed}{(}mask\textcolor{BrickRed}{[}j\textcolor{BrickRed}{]+}\textcolor{Purple}{1}\textcolor{BrickRed}{)\%\%}\textcolor{Purple}{2} \\
\mbox{}\ \ \ \ \textbf{\textcolor{Blue}{return}}\ mask \\
\mbox{}\textcolor{Red}{\}} \\
\mbox{}
}
  \normalsize
  
  
  El algoritmo, propiamente dicho, que se ha empleado ha sido:\\
  
  \small\texttt{% Generator: GNU source-highlight, by Lorenzo Bettini, http://www.gnu.org/software/src-highlite
\noindent
\mbox{}\textbf{\textcolor{Blue}{def}}\ BL\textcolor{BrickRed}{(}m\textcolor{BrickRed}{)}\textcolor{Red}{\{} \\
\mbox{}\ \ \ \ \textit{\textcolor{Brown}{\#\ Máscara\ inicial,\ por\ defecto\ aleatoria}} \\
\mbox{}\ \ \ \ mascara\ \textcolor{BrickRed}{=}\ m \\
\mbox{}\ \ \ \ no$\_$seleccionados\ \textcolor{BrickRed}{=}\ \textcolor{Red}{\{}\textcolor{Purple}{1}\textcolor{BrickRed}{,}\textcolor{Purple}{2}\textcolor{BrickRed}{,...}n\textcolor{Red}{\}} \\
\mbox{} \\
\mbox{}\ \ \ \ \textbf{\textcolor{Blue}{do}}\textcolor{Red}{\{} \\
\mbox{}\ \ \ \ \ \ \ \ i\ \textcolor{BrickRed}{=}\ random\textcolor{BrickRed}{(}no$\_$seleccionados\textcolor{BrickRed}{)} \\
\mbox{}\ \ \ \ \ \ \ \ no$\_$seleccionados\ \textcolor{BrickRed}{=}\ no$\_$seleccionados\ \textcolor{BrickRed}{-}\textcolor{Red}{\{}i\textcolor{Red}{\}} \\
\mbox{}\ \ \ \ \ \ \ \ evs$\_$hechas\textcolor{BrickRed}{++} \\
\mbox{} \\
\mbox{}\ \ \ \ \ \ \ \ \textbf{\textcolor{Blue}{if}}\ \textcolor{BrickRed}{(}tasa\textcolor{BrickRed}{(}flip\textcolor{BrickRed}{(}mascara\textcolor{BrickRed}{,}\ i\textcolor{BrickRed}{)}\ \textcolor{BrickRed}{\textgreater{}}\ mejor$\_$tasa\textcolor{BrickRed}{)}\textcolor{Red}{\{} \\
\mbox{}\ \ \ \ \ \ \ \ \ \ \ \ mascara\ \textcolor{BrickRed}{=}\ flip\textcolor{BrickRed}{(}mascara\textcolor{BrickRed}{,}i\textcolor{BrickRed}{)} \\
\mbox{}\ \ \ \ \ \ \ \ \ \ \ \ mejor$\_$tasa\ \textcolor{BrickRed}{=}\ tasa\textcolor{BrickRed}{(}mascara\textcolor{BrickRed}{)} \\
\mbox{}\ \ \ \ \ \ \ \ \ \ \ \ mejora$\_$encontrada\ \textcolor{BrickRed}{=}\ \textbf{\textcolor{Blue}{true}} \\
\mbox{}\ \ \ \ \ \ \ \ \textcolor{Red}{\}} \\
\mbox{}\ \ \ \ \textcolor{Red}{\}}\textbf{\textcolor{Blue}{while}}\ \textcolor{BrickRed}{!}no$\_$seleccionados\textcolor{BrickRed}{.}empty\ \textbf{\textcolor{Blue}{and}}\ \textcolor{BrickRed}{!}mejora$\_$encontrado \\
\mbox{} \\
\mbox{}\ \ \textbf{\textcolor{Blue}{return}}\ \textcolor{BrickRed}{(}mascara\textcolor{BrickRed}{,}\ evs$\_$hechas\textcolor{BrickRed}{)} \\
\mbox{}\textcolor{Red}{\}} \\
\mbox{}
}
  \normalsize
  
  Se realiza 1 iteracion en el bucle principal de la búsqueda local, tanto si se ha encontrado mejora en el entorno como
  si no. Como solución se devuelve la máscara junto al número de evaluaciones de la función objetivo efectuado.

  \item \textbf{Actualización de la feromona}:
  \footnote{Se introduce este aspecto en este apartado y no el siguiente dado que el lenguaje de programación empleado,
  R, ha posibilitado emplear la misma función para ambos algoritmos}
  
  \small\texttt{% Generator: GNU source-highlight, by Lorenzo Bettini, http://www.gnu.org/software/src-highlite
\noindent
\mbox{} \\
\mbox{}\textbf{\textcolor{Blue}{def}}\ update$\_$trail\textcolor{BrickRed}{(}trail\textcolor{BrickRed}{,}\ factor$\_$evap\textcolor{BrickRed}{,}\ extra\textcolor{BrickRed}{,}\ mask\textcolor{BrickRed}{)}\textcolor{Red}{\{} \\
\mbox{}\ \ \ \ trail\ \textcolor{BrickRed}{=}\ \textcolor{BrickRed}{(}trail\ \textcolor{BrickRed}{*}\ \textcolor{BrickRed}{!}mask\ \textcolor{BrickRed}{+} \\
\mbox{}\ \ \ \ \ \ \ \ \textcolor{BrickRed}{(}\ \textcolor{BrickRed}{(}\textcolor{Purple}{1}\ \textcolor{BrickRed}{-}\ factor$\_$evap\textcolor{BrickRed}{)}\ \textcolor{BrickRed}{*}\ trail\ \textcolor{BrickRed}{+}\ factor$\_$evap\ \textcolor{BrickRed}{*}\ extra\textcolor{BrickRed}{)}\ \textcolor{BrickRed}{*}\ mask\textcolor{BrickRed}{)} \\
\mbox{}\textcolor{Red}{\}} \\
\mbox{}
}
  \normalsize
  
\end{itemize}

\end{itemize}
%\newpage
\section{Profundización en los algoritmos}
Se ha empleado una metafunción a la que pasándole distintos \textit{settings} es capaz de efectuar los dos algoritmos.
Por claridad en este apartado se detallan los pseudocódigos de ambos algoritmos por separado, aunque se puede apreciar que
las únicas diferencias notables entre ambos las constituyen la actualización local de feromona, que en SHMM no está presente,
y el tratamiento de $\tau_{max}$ y $\tau_{min}$, que solo se encuentra en SHMM.


  \subsection{Algoritmo de Colonia de Hormigas}.

  \small{\texttt{% Generator: GNU source-highlight, by Lorenzo Bettini, http://www.gnu.org/software/src-highlite
\noindent
\mbox{}\textbf{\textcolor{Blue}{def}}\ SCH$\_$BL\textcolor{BrickRed}{(}data\textcolor{BrickRed}{)}\textcolor{Red}{\{} \\
\mbox{}\ \ \ \ data$\_$heuristic\ \textcolor{BrickRed}{=}\ entropy\textcolor{BrickRed}{(}data\textcolor{BrickRed}{)} \\
\mbox{} \\
\mbox{}\ \ \ \ \textbf{\textcolor{Blue}{while}}\textcolor{BrickRed}{(}n$\_$eval\ \textcolor{BrickRed}{\textless{}}\ max$\_$eval\textcolor{BrickRed}{)}\textcolor{Red}{\{} \\
\mbox{}\ \ \ \ \ \ \ \ init$\_$iteration\textcolor{BrickRed}{(}data\textcolor{BrickRed}{)} \\
\mbox{} \\
\mbox{}\ \ \ \ \ \ \ \ \textbf{\textcolor{Blue}{for}}\ k\ \textbf{\textcolor{Blue}{in}}\ \textcolor{Red}{\{}\textcolor{Purple}{1}\textcolor{BrickRed}{...}num\textcolor{BrickRed}{.}ants\textcolor{Red}{\}\{} \\
\mbox{}\ \ \ \ \ \ \ \ \ \ \ \ paths\textcolor{BrickRed}{[}k\textcolor{BrickRed}{]}\ \textcolor{BrickRed}{=}\ make\textcolor{BrickRed}{.}transitions\textcolor{BrickRed}{(}paths\textcolor{BrickRed}{[[}k\textcolor{BrickRed}{]],}\ num$\_$features\textcolor{BrickRed}{[}k\textcolor{BrickRed}{],}\ trail$\_$features\textcolor{BrickRed}{)} \\
\mbox{} \\
\mbox{}\ \ \ \ \ \ \ \ \ \ \ \ \textit{\textcolor{Brown}{\#\ Actualizacion\ local\ de\ la\ feromona}} \\
\mbox{}\ \ \ \ \ \ \ \ \ \ \ \ update$\_$trail\textcolor{BrickRed}{(}\ \ trail$\_$features\textcolor{BrickRed}{,}\ local$\_$evap\textcolor{BrickRed}{,}\ \textcolor{Purple}{10}\textcolor{BrickRed}{\textasciicircum{}(-}\textcolor{Purple}{6}\textcolor{BrickRed}{),}\ paths\textcolor{BrickRed}{[}k\textcolor{BrickRed}{]}\ \textcolor{BrickRed}{)} \\
\mbox{}\ \ \ \ \ \ \ \ \textcolor{Red}{\}} \\
\mbox{} \\
\mbox{}\ \ \ \ \ \ \ \ \textit{\textcolor{Brown}{\#\ Aplicamos\ busqueda\ local}} \\
\mbox{}\ \ \ \ \ \ \ \ paths\textcolor{BrickRed}{,}\ evs$\_$BL\ \textcolor{BrickRed}{=}\ map\textcolor{BrickRed}{(}BL\textcolor{BrickRed}{,}\ paths\textcolor{BrickRed}{)} \\
\mbox{} \\
\mbox{}\ \ \ \ \ \ \ \ scores\ \textcolor{BrickRed}{=}\ map\textcolor{BrickRed}{(}tasa$\_$clasificacion\textcolor{BrickRed}{,}\ paths\textcolor{BrickRed}{)} \\
\mbox{}\ \ \ \ \ \ \ \ n$\_$eval\ \textcolor{BrickRed}{=}\ n$\_$eval\ \textcolor{BrickRed}{+}\ sum\textcolor{BrickRed}{(}evs$\_$BL\textcolor{BrickRed}{)}\ \textcolor{BrickRed}{+}\ num$\_$ants \\
\mbox{} \\
\mbox{}\ \ \ \ \ \ \ \ \textbf{\textcolor{Blue}{if}}\textcolor{BrickRed}{(}max\textcolor{BrickRed}{(}scores\textcolor{BrickRed}{)}\ \textcolor{BrickRed}{\textgreater{}}\ tasa$\_$best\textcolor{BrickRed}{)} \\
\mbox{}\ \ \ \ \ \ \ \ \ \ \ \ mask$\_$best\ \textcolor{BrickRed}{=}\ paths\textcolor{BrickRed}{[}\ arg$\_$max\textcolor{BrickRed}{(}scores\textcolor{BrickRed}{)}\ \textcolor{BrickRed}{]} \\
\mbox{} \\
\mbox{}\ \ \ \ \ \ \ \ \textit{\textcolor{Brown}{\#\ Actualizacion\ global\ de\ feromona}} \\
\mbox{}\ \ \ \ \ \ \ \ update$\_$trail\textcolor{BrickRed}{(}\ trail$\_$features\textcolor{BrickRed}{,}\ global$\_$evap\textcolor{BrickRed}{,} \\
\mbox{}\ \ \ \ \ \ \ \ \ \ \ \ \ \ \ \ \ \ \ \ \ \ scores\textcolor{BrickRed}{[}arg$\_$max\textcolor{BrickRed}{(}scores\textcolor{BrickRed}{)],}\ paths\textcolor{BrickRed}{[}arg$\_$max\textcolor{BrickRed}{(}scores\textcolor{BrickRed}{)]}\ \textcolor{BrickRed}{)} \\
\mbox{}\ \ \ \ \ \ \ \ \textit{\textcolor{Brown}{\#\ El\ único\ 1\ del\ vector\ que\ pasamos\ a\ update$\_$trail\ está\ en\ arg$\_$max(scores)}} \\
\mbox{}\ \ \ \ \ \ \ \ update\textcolor{BrickRed}{.}trail\textcolor{BrickRed}{(}trail$\_$num$\_$features\textcolor{BrickRed}{,}\ global$\_$evap\textcolor{BrickRed}{,}\ max\textcolor{BrickRed}{(}score\textcolor{BrickRed}{),}\ \textcolor{BrickRed}{[}\ \textcolor{Purple}{0}\textcolor{BrickRed}{,}\textcolor{Purple}{0}\textcolor{BrickRed}{,...}\textcolor{Purple}{1}\textcolor{BrickRed}{,...}\textcolor{Purple}{0}\textcolor{BrickRed}{])} \\
\mbox{}\ \ \ \ \textcolor{Red}{\}} \\
\mbox{} \\
\mbox{}\ \ \ \ \textbf{\textcolor{Blue}{return}}\ mask$\_$best \\
\mbox{}\textcolor{Red}{\}} \\
\mbox{}
}}
  \normalsize

  \subsection{Algoritmo de Sistema de Hormigas Max Min}

  \small{\texttt{% Generator: GNU source-highlight, by Lorenzo Bettini, http://www.gnu.org/software/src-highlite
\noindent
\mbox{}\textbf{\textcolor{Blue}{def}}\ SHMM$\_$BL\textcolor{BrickRed}{(}data\textcolor{BrickRed}{)}\textcolor{Red}{\{} \\
\mbox{}\ \ \ \ data$\_$heuristic\ \textcolor{BrickRed}{=}\ entropy\textcolor{BrickRed}{(}data\textcolor{BrickRed}{)} \\
\mbox{} \\
\mbox{}\ \ \ \ \textbf{\textcolor{Blue}{while}}\textcolor{BrickRed}{(}n$\_$eval\ \textcolor{BrickRed}{\textless{}}\ max$\_$eval\textcolor{BrickRed}{)}\textcolor{Red}{\{} \\
\mbox{}\ \ \ \ \ \ \ \ init$\_$iteration\textcolor{BrickRed}{(}data\textcolor{BrickRed}{)} \\
\mbox{} \\
\mbox{}\ \ \ \ \ \ \ \ \textbf{\textcolor{Blue}{for}}\ k\ \textbf{\textcolor{Blue}{in}}\ \textcolor{Red}{\{}\textcolor{Purple}{1}\textcolor{BrickRed}{...}num\textcolor{BrickRed}{.}ants\textcolor{Red}{\}\{} \\
\mbox{}\ \ \ \ \ \ \ \ \ \ \ \ paths\textcolor{BrickRed}{[}k\textcolor{BrickRed}{]}\ \textcolor{BrickRed}{=}\ make$\_$transitions\textcolor{BrickRed}{(}paths\textcolor{BrickRed}{[[}k\textcolor{BrickRed}{]],}\ num$\_$features\textcolor{BrickRed}{[}k\textcolor{BrickRed}{],}\ trail$\_$features\textcolor{BrickRed}{)} \\
\mbox{}\ \ \ \ \ \ \ \ \textcolor{Red}{\}} \\
\mbox{} \\
\mbox{}\ \ \ \ \ \ \ \ \textit{\textcolor{Brown}{\#\ Aplicamos\ busqueda\ local}} \\
\mbox{}\ \ \ \ \ \ \ \ paths\textcolor{BrickRed}{,}\ evs$\_$BL\ \textcolor{BrickRed}{=}\ map\textcolor{BrickRed}{(}BL\textcolor{BrickRed}{,}\ paths\textcolor{BrickRed}{)} \\
\mbox{} \\
\mbox{}\ \ \ \ \ \ \ \ scores\ \textcolor{BrickRed}{=}\ map\textcolor{BrickRed}{(}tasa\textcolor{BrickRed}{,}\ paths\textcolor{BrickRed}{)} \\
\mbox{}\ \ \ \ \ \ \ \ n$\_$eval\ \textcolor{BrickRed}{=}\ n$\_$eval\ \textcolor{BrickRed}{+}\ sum\textcolor{BrickRed}{(}evs$\_$BL\textcolor{BrickRed}{)}\ \textcolor{BrickRed}{+}\ num$\_$ants \\
\mbox{} \\
\mbox{}\ \ \ \ \ \ \ \ \textbf{\textcolor{Blue}{if}}\textcolor{BrickRed}{(}max\textcolor{BrickRed}{(}scores\textcolor{BrickRed}{)}\ \textcolor{BrickRed}{\textgreater{}}\ tasa\textcolor{BrickRed}{(}mask$\_$best\textcolor{BrickRed}{))}\textcolor{Red}{\{} \\
\mbox{}\ \ \ \ \ \ \ \ \ \ \ \ mask$\_$best\ \textcolor{BrickRed}{=}\ paths\textcolor{BrickRed}{[}\ arg$\_$max\textcolor{BrickRed}{(}scores\textcolor{BrickRed}{)}\ \textcolor{BrickRed}{]} \\
\mbox{}\ \ \ \ \ \ \ \ \ \ \ \ trail$\_$max\ \textcolor{BrickRed}{=}\ tasa\textcolor{BrickRed}{(}\ mask$\_$best\ \textcolor{BrickRed}{)} \\
\mbox{}\ \ \ \ \ \ \ \ \ \ \ \ trail$\_$min\ \textcolor{BrickRed}{=}\ trail$\_$max\textcolor{BrickRed}{/}\textcolor{Purple}{500} \\
\mbox{}\ \ \ \ \ \ \ \ \textcolor{Red}{\}} \\
\mbox{}\ \ \ \ \ \ \ \ \textit{\textcolor{Brown}{\#\ Actualizacion\ global\ de\ feromona}} \\
\mbox{}\ \ \ \ \ \ \ \ update$\_$trail\textcolor{BrickRed}{(}\ trail$\_$features\textcolor{BrickRed}{,}\ global$\_$evap\textcolor{BrickRed}{,} \\
\mbox{}\ \ \ \ \ \ \ \ \ \ \ \ \ \ \ \ \ \ \ \ \ \ scores\textcolor{BrickRed}{[}arg$\_$max\textcolor{BrickRed}{(}scores\textcolor{BrickRed}{)],}\ paths\textcolor{BrickRed}{[}arg$\_$max\textcolor{BrickRed}{(}scores\textcolor{BrickRed}{)]}\ \textcolor{BrickRed}{)} \\
\mbox{}\ \ \ \ \ \ \ \ \textit{\textcolor{Brown}{\#\ El\ único\ 1\ del\ vector\ que\ pasamos\ a\ update$\_$trail\ está\ en\ arg$\_$max(scores)}} \\
\mbox{}\ \ \ \ \ \ \ \ update$\_$trail\textcolor{BrickRed}{(}trail$\_$num$\_$features\textcolor{BrickRed}{,}\ global$\_$evap\textcolor{BrickRed}{,}\ max\textcolor{BrickRed}{(}score\textcolor{BrickRed}{),}\ \textcolor{BrickRed}{[}\textcolor{Purple}{0}\textcolor{BrickRed}{,}\textcolor{Purple}{0}\textcolor{BrickRed}{,...}\textcolor{Purple}{1}\textcolor{BrickRed}{,...}\textcolor{Purple}{0}\textcolor{BrickRed}{])} \\
\mbox{} \\
\mbox{} \\
\mbox{}\ \ \ \ \ \ \ \ trail$\_$features\textcolor{BrickRed}{[}\ which\textcolor{BrickRed}{(}trail$\_$features\ \textcolor{BrickRed}{\textgreater{}}\ trail$\_$max\textcolor{BrickRed}{)}\ \textcolor{BrickRed}{]}\ \textcolor{BrickRed}{=}\ trail$\_$max \\
\mbox{}\ \ \ \ \ \ \ \ trail$\_$features\textcolor{BrickRed}{[}\ which\textcolor{BrickRed}{(}trail$\_$features\ \textcolor{BrickRed}{\textless{}}\ trail$\_$min\textcolor{BrickRed}{)}\ \textcolor{BrickRed}{]}\ \textcolor{BrickRed}{=}\ trail$\_$min \\
\mbox{}\ \ \ \ \textcolor{Red}{\}} \\
\mbox{}\ \ \ \ \textbf{\textcolor{Blue}{return}}\ mask$\_$best \\
\mbox{}\textcolor{Red}{\}} \\
\mbox{}
}}
  \normalsize

  
\section{Algoritmo de comparación: SFS}
\small{\texttt{% Generator: GNU source-highlight, by Lorenzo Bettini, http://www.gnu.org/software/src-highlite
\noindent
\mbox{}\textbf{\textcolor{Blue}{def}}\ SFS\textcolor{BrickRed}{()}\textcolor{Red}{\{} \\
\mbox{}\ \ \ \ non$\_$selected\ \textcolor{BrickRed}{=}\ \textcolor{Red}{\{}\textcolor{Purple}{1}\textcolor{BrickRed}{,...}n\textcolor{Red}{\}} \\
\mbox{}\ \ \ \ mascara\ \textcolor{BrickRed}{=}\ \textcolor{Red}{\{}\textcolor{Purple}{0}\textcolor{BrickRed}{,}\textcolor{Purple}{0}\textcolor{BrickRed}{...}\textcolor{Purple}{0}\textcolor{Red}{\}} \\
\mbox{}\ \ \ \ mejora\ \textcolor{BrickRed}{=}\ \textbf{\textcolor{Blue}{true}} \\
\mbox{} \\
\mbox{}\ \ \ \ \textbf{\textcolor{Blue}{do}}\textcolor{Red}{\{} \\
\mbox{}\ \ \ \ \ \ \ \ j\ \textcolor{BrickRed}{=}\ max\ arg\textcolor{Red}{\{}tasa\textcolor{BrickRed}{(}flip\textcolor{BrickRed}{(}mascara\textcolor{BrickRed}{,}x\textcolor{BrickRed}{)):}\ x\$\textcolor{BrickRed}{\textbackslash{}}\textbf{\textcolor{Blue}{in}}\$\ non$\_$selected\textcolor{Red}{\}} \\
\mbox{} \\
\mbox{}\ \ \ \ \ \ \ \ \textbf{\textcolor{Blue}{if}}\ \textcolor{BrickRed}{(}tasa\textcolor{BrickRed}{(}flip\textcolor{BrickRed}{(}mascara\textcolor{BrickRed}{,}j\textcolor{BrickRed}{))}\ \textcolor{BrickRed}{\textgreater{}}\ tasa\textcolor{BrickRed}{(}mascara\textcolor{BrickRed}{))}\textcolor{Red}{\{} \\
\mbox{}\ \ \ \ \ \ \ \ \ \ \ \ non$\_$selected\ \textcolor{BrickRed}{=}\ non$\_$selected\ \textcolor{BrickRed}{-}\ \textcolor{Red}{\{}j\textcolor{Red}{\}} \\
\mbox{}\ \ \ \ \ \ \ \ \ \ \ \ mascara\ \textcolor{BrickRed}{[}j\textcolor{BrickRed}{]}\ \textcolor{BrickRed}{=}\ \textcolor{Purple}{1} \\
\mbox{}\ \ \ \ \ \ \ \ \textcolor{Red}{\}} \\
\mbox{}\ \ \ \ \ \ \ \ \textbf{\textcolor{Blue}{else}} \\
\mbox{}\ \ \ \ \ \ \ \ \ \ \ \ mejora\ \textcolor{BrickRed}{=}\ \textbf{\textcolor{Blue}{false}} \\
\mbox{}\ \ \ \ \textcolor{Red}{\}}\textbf{\textcolor{Blue}{while}}\ mejora \\
\mbox{} \\
\mbox{}\ \ \textbf{\textcolor{Blue}{return}}\ mascara \\
\mbox{}\textcolor{Red}{\}} \\
\mbox{}
}}

Partiendo de una selección de características vacía (máscara nula), va añadiendo a cada paso la característica
de las no escogidas hasta el momento que maximiza la tasa de clasificación añadiéndola a la solución, hasta
que la característica escogida no mejore a la mejor solución hasta el momento

\section{Implementación de la práctica}
Para la implementación de la práctica, se ha reutilizado toda la estructura de lectura de ficheros y funciones auxiliares
empleada para las dos anteriores.

El clasificador \texttt{3nn} leaving one out empleado es el incluido en el paquete \texttt{class}, de nombre
\texttt{knn.cv}. Para su uso, se le pasa como argumento el número de vecinos a considerar (3). El argumento 
\texttt{use.all=TRUE} indica que en caso de empate considere todas las etiquetas de las instancias con distancias 
iguales a la mayoritaria (en este caso las tres), y escoja la etiqueta mayoritaria de entre todas. 

Se hace uso de la función (\texttt{normalize(data.frame)}) que limpia los \textit{datasets} de columnas con
un único valor, establece los nombres de los atributos todos a minúsculas, reordena los atributos del dataset para que
la clase sea el último de todos, y hace una transformación con las columnas para que todos los valores estén comprendidos
entre 0 y 1.

Se describe a continuación un esquema de los ficheros de código:
\begin{itemize}
 \item \hrefr{main.r}: de arriba a abajo, carga los paquetes necesarios, incluye el código fuente de otros ficheros
  (\hrefr{aux.r}, \hrefr{AGs.r}, \ldots) y lee los parámetros necesarios, entre ellos las 
  semillas aleatorias (12345678, 23456781, 34567812, 45678123, 56781234) desde el fichero \hrefr{params.r}
  
  Se ejecuta la función \texttt{cross.eval(algoritmo)}, que almacena los resultados de la ejecución en la lista
  \texttt{algoritmo.results}. Esta lista contendrá 3 dataframes con los datos del 5x2 cross validation, uno por dataset, 
  con 10 filas cada uno  y columnas las tasas de clasificación de test y train, la tasa de reducción y el tiempo de 
  ejecución. Incluye también 3 datasets más con la media de los datos anteriores para cada conjunto de datos.
 
  Esta función se ejecutará pasándole como parámetros la función del algoritmo genético generacional y la del estacionario.
  
 \item \hrefr{aux.r}: contiene las implementaciones de la función de normalización de datasets, la función de
 particionado de \texttt{data.frames}, la función objetivo \texttt{tasa.clas} y la función \texttt{cross.eval}
 descrita arriba.
 
 \item \hrefr{AGs.r}: contiene:
  \begin{itemize}
   \item La función \texttt{random.init} que se aplica 30 veces en cada algoritmo para inicializar la
 población con máscaras aleatorias
   \item La función \texttt{crossover.OX} que es el operador de cruce que por defecto se usa. Podría implementarse otro y
   aplicarlo en lugar del OX de forma fácil según la estructura que se ha creado al programar.
   \item Una metafunción de nombre AG, a la que se le pasa un dataset y un operador de cruce, y devuelve dos funciones,
   una que al ejecutarla realiza el proceso de algoritmo generacional, y otra del estacionario. Se ha optado por esta 
   técnica de implementación para encapsular todo lo común a los dos algoritmos, que como se ha descrito en el presente
   documento se han implementado usando las mismas funciones auxiliares para efectuar selecciones-torneos, cruces, mutaciones,
   etc.
  \end{itemize}
 
 \item \hrefr{params.r}: fichero del que se leen los parámetros:
  \begin{itemize}
    \item \texttt{semilla}: vector de semillas aleatorias para poder reproducir los análisis hechos.
    \item \texttt{max\_eval}: número máximo de evaluaciones de la función objetivo.
    \item \texttt{AG.n.crom}: tamaño de la población. Por defecto seteado a 30.
    \item \texttt{AGG.prob.cruce, AGE.prob.cruce}: probabilidad de cruce de los padres para dar lugar a un nuevo cromosoma 
    en los algoritmos generacional y estacionario, respectivamente. Está seteada a 0.7 y a 1 por defecto.
    \item \texttt{AGG.prob.cruce, AGE.prob.cruce}: probabilidad de mutación de los genes(cada una de las $n$ características
    de la máscara) en las soluciones del generacional y estacionario, respectivamente. Está seteada a 0.001 (1 gen por cada mil)
  \end{itemize}
 \end{itemize}
 
 Para verificar los datos aportados para el algoritmo \texttt{algx}, a saber, basta ejecutar todo el fichero \texttt{main.r}
 hasta justo antes de la sección \textit{Obtención de resultados}. A continuación, si se ejecuta la línea 
 \texttt{algx.results <- cross.eval(algx)} se obtienen almacenados en una lista los valores de tasas y tiempos de ejecución
 aportados en la presente memoria. 
 
 Para trabajar correctamente, el working path debe estar seteado a la carpeta de códigos fuente. Se puede consultar el 
 valor actual mediante \texttt{getwd()} y setearlo mediante \texttt{setwd(path)}
 
 Cuando se modifica algo en algún algoritmo o en \texttt{params.r} hay que recargar en \texttt{main.r}
 la línea del \texttt{source(file=...)} correspondiente.
 
 \section{Experimentos y análisis de resultados}
 \subsection{Tablas de resultados}
  \begin{table}[H]	
  \caption{Resultados del 3NN}
  \begin{adjustbox}{width=1.1\textwidth}
  \begin{tabular}{|c|r|r|r|r|r|r|r|r|r|r|r|r|}
  \hline
  \multicolumn{1}{|l|}{} & \multicolumn{ 4}{c|}{\textbf{\textit{Wdbc}}} & \multicolumn{ 4}{c|}{\textbf{\textit{Movement\_Libras}}} & \multicolumn{ 4}{c|}{\textbf{\textit{Arrhythmia}}} \\ \hline
  & \multicolumn{1}{c|}{\textbf{\textit{\% clas test}}} & \multicolumn{1}{c|}{\textbf{\textit{\% clas train}}} & \multicolumn{1}{c|}{\textbf{\textit{tasa red}}} & \multicolumn{1}{c|}{\textbf{T(s)}} & \multicolumn{1}{c|}{\textbf{\textit{\% clas test}}} & \multicolumn{1}{c|}{\textbf{\textit{\% clas train}}} & \multicolumn{1}{c|}{\textbf{\textit{tasa red}}} & \multicolumn{1}{c|}{\textbf{T(s)}} & \multicolumn{1}{c|}{\textbf{\textit{\% clas test}}} & \multicolumn{1}{c|}{\textbf{\textit{\% clas train}}} & \multicolumn{1}{c|}{\textbf{\textit{tasa red}}} & \multicolumn{1}{c|}{\textbf{T(s)}} \\ \hline
  \textbf{Partición 1-1} & 95.78947 & 97.53521 & 0.00000 & 0.00000 & 65.00000 & 66.66667 & 0.00000 & 0.00000 & 65.46392 & 65.62500 & 0.00000 & 0.00000 \\ \hline
  \textbf{Partición 1-2} & 97.53521 & 95.78947 & 0.00000 & 0.00000 & 66.11111 & 67.77778 & 0.00000 & 0.00000 & 65.62500 & 65.97938 & 0.00000 & 0.00000 \\ \hline
  \textbf{Partición 2-1} & 97.19298 & 95.77465 & 0.00000 & 0.00000 & 72.22222 & 64.44444 & 0.00000 & 0.00000 & 62.88660 & 61.45833 & 0.00000 & 0.00000 \\ \hline
  \textbf{Partición 2-2} & 95.77465 & 97.19298 & 0.00000 & 0.00000 & 63.33333 & 70.00000 & 0.00000 & 0.00000 & 63.02083 & 63.91753 & 0.00000 & 0.00000 \\ \hline
  \textbf{Partición 3-1} & 96.14035 & 96.47887 & 0.00000 & 0.00000 & 72.77778 & 65.00000 & 0.00000 & 0.00000 & 62.37113 & 64.06250 & 0.00000 & 0.00000 \\ \hline
  \textbf{Partición 3-2} & 96.47887 & 96.14035 & 0.00000 & 0.00000 & 63.33333 & 75.00000 & 0.00000 & 0.00000 & 63.54167 & 62.88660 & 0.00000 & 0.00000 \\ \hline
  \textbf{Partición 4-1} & 95.43860 & 97.88732 & 0.00000 & 0.00000 & 74.44444 & 66.66667 & 0.00000 & 0.00000 & 64.94845 & 62.50000 & 0.00000 & 0.00000 \\ \hline
  \textbf{Partición 4-2} & 97.88732 & 95.43860 & 0.00000 & 0.00000 & 64.44444 & 72.77778 & 0.00000 & 0.00000 & 61.45833 & 62.88660 & 0.00000 & 0.00000 \\ \hline
  \textbf{Partición 5-1} & 96.49123 & 96.83099 & 0.00000 & 0.00000 & 63.33333 & 68.33333 & 0.00000 & 0.00000 & 61.85567 & 61.45833 & 0.00000 & 0.00000 \\ \hline
  \textbf{Partición 5-2} & 96.83099 & 96.49123 & 0.00000 & 0.00000 & 67.77778 & 65.55556 & 0.00000 & 0.00000 & 60.41667 & 62.37113 & 0.00000 & 0.00000 \\ \hline
  \textbf{Media} & 96.55597 & 96.55597 & 0.00000 & 0.00000 & 67.27778 & 68.22222 & 0.00000 & 0.00000 & 63.15883 & 63.31454 & 0.00000 & 0.00000 \\ \hline
  \textbf{Desv.Típica} & 0.76434 & 0.76434 & 0.00000 & 0.00000 & 4.09343 & 3.26599 & 0.00000 & 0.00000 & 1.66035 & 1.48907 & 0.00000 & 0.00000 \\ \hline
  \end{tabular}
  \end{adjustbox}
  \label{NN3}
  \end{table}
  
  \begin{table}[H]	
  \caption{Resultados del SFS}
  \begin{adjustbox}{width=1.1\textwidth}
  \begin{tabular}{|c|r|r|r|r|r|r|r|r|r|r|r|r|}
  \hline
  \multicolumn{1}{|l|}{} & \multicolumn{ 4}{c|}{\textbf{\textit{Wdbc}}} & \multicolumn{ 4}{c|}{\textbf{\textit{Movement\_Libras}}} & \multicolumn{ 4}{c|}{\textbf{\textit{Arrhytmia}}} \\ \hline
  \multicolumn{1}{|l|}{} & \multicolumn{1}{c|}{\textbf{\textit{\% clas test}}} & \multicolumn{1}{c|}{\textbf{\textit{\% clas train}}} & \multicolumn{1}{c|}{\textbf{\textit{tasa red}}} & \multicolumn{1}{c|}{\textbf{T(s)}} & \multicolumn{1}{c|}{\textbf{\textit{\% clas test}}} & \multicolumn{1}{c|}{\textbf{\textit{\% clas train}}} & \multicolumn{1}{c|}{\textbf{\textit{tasa red}}} & \multicolumn{1}{c|}{\textbf{T(s)}} & \multicolumn{1}{c|}{\textbf{\textit{\% clas test}}} & \multicolumn{1}{c|}{\textbf{\textit{\% clas train}}} & \multicolumn{1}{c|}{\textbf{\textit{tasa red}}} & \multicolumn{1}{c|}{\textbf{T(s)}} \\ \hline
  \textbf{Partición 1-1} & 91.92982 & 97.53521 & 0.86667 & 0.15400 & 63.88889 & 70.55556 & 0.88889 & 1.01200 & 64.94845 & 75.00000 & 0.98419 & 1.84000 \\ \hline
  \textbf{Partición 1-2} & 95.77465 & 97.89474 & 0.80000 & 0.25600 & 65.00000 & 68.33333 & 0.87778 & 1.29800 & 75.00000 & 78.35052 & 0.98419 & 1.98900 \\ \hline
  \textbf{Partición 2-1} & 97.19298 & 97.88732 & 0.86667 & 0.15200 & 64.44444 & 66.66667 & 0.93333 & 0.57300 & 64.94845 & 76.56250 & 0.98419 & 1.95300 \\ \hline
  \textbf{Partición 2-2} & 94.01408 & 97.19298 & 0.80000 & 0.26600 & 62.22222 & 69.44444 & 0.87778 & 1.13400 & 73.43750 & 71.13402 & 0.98419 & 2.22600 \\ \hline
  \textbf{Partición 3-1} & 94.38596 & 95.77465 & 0.93333 & 0.09700 & 66.66667 & 66.11111 & 0.88889 & 1.02800 & 72.16495 & 80.20833 & 0.98024 & 2.34900 \\ \hline
  \textbf{Partición 3-2} & 91.54930 & 96.14035 & 0.90000 & 0.14500 & 62.22222 & 74.44444 & 0.88889 & 1.20200 & 67.70833 & 74.22680 & 0.98024 & 2.20000 \\ \hline
  \textbf{Partición 4-1} & 94.03509 & 97.88732 & 0.90000 & 0.12000 & 71.66667 & 72.22222 & 0.88889 & 1.31100 & 74.22680 & 76.04167 & 0.98024 & 2.02000 \\ \hline
  \textbf{Partición 4-2} & 92.25352 & 94.38596 & 0.90000 & 0.11700 & 65.55556 & 77.22222 & 0.90000 & 0.87400 & 67.70833 & 76.80412 & 0.98419 & 2.44800 \\ \hline
  \textbf{Partición 5-1} & 94.73684 & 95.77465 & 0.86667 & 0.15500 & 58.33333 & 75.55556 & 0.91111 & 0.76600 & 67.52577 & 75.52083 & 0.98419 & 2.30900 \\ \hline
  \textbf{Partición 5-2} & 90.49296 & 96.49123 & 0.86667 & 0.15300 & 65.00000 & 67.22222 & 0.90000 & 0.99900 & 70.83333 & 74.74227 & 0.98814 & 1.33000 \\ \hline
  \textbf{Media} & 93.63652 & 96.69644 & 0.87000 & 0.16150 & 64.50000 & 70.77778 & 0.89556 & 1.01970 & 69.85019 & 75.85911 & 0.98340 & 2.06640 \\ \hline
  \textbf{Desv.Típica} & 1.95851 & 1.12301 & 0.04069 & 0.05315 & 3.26268 & 3.72844 & 0.01587 & 0.22233 & 3.57073 & 2.31592 & 0.00237 & 0.30695 \\ \hline
  \end{tabular}
  \end{adjustbox}
  \label{SFS}
  \end{table}
  
 
 \begin{table}[H]	
  \caption{Resultados del AGG}
  \begin{adjustbox}{width=1.1\textwidth}
  \begin{tabular}{|c|r|r|r|r|r|r|r|r|r|r|r|r|}
  \hline
  \multicolumn{1}{|l|}{} & \multicolumn{1}{c|}{\textbf{\textit{\% clas test}}} & \multicolumn{1}{c|}{\textbf{\textit{\% clas train}}} & \multicolumn{1}{c|}{\textbf{\textit{tasa red}}} & \multicolumn{1}{c|}{\textbf{T(s)}} & \multicolumn{1}{c|}{\textbf{\textit{\% clas test}}} & \multicolumn{1}{c|}{\textbf{\textit{\% clas train}}} & \multicolumn{1}{c|}{\textbf{\textit{tasa red}}} & \multicolumn{1}{c|}{\textbf{T(s)}} & \multicolumn{1}{c|}{\textbf{\textit{\% clas test}}} & \multicolumn{1}{c|}{\textbf{\textit{\% clas train}}} & \multicolumn{1}{c|}{\textbf{\textit{tasa red}}} & \multicolumn{1}{c|}{\textbf{T(s)}} \\ \hline
  \textbf{Partición 1-1} & 96.14035 & 98.94366 & 0.46667 & 42.45600 & 63.88889 & 70.55556 & 0.53333 & 69.70300 & 69.07216 & 68.22917 & 0.47431 & 540.86500 \\ \hline
  \textbf{Partición 1-2} & 95.07042 & 98.59649 & 0.63333 & 43.79400 & 65.55556 & 71.11111 & 0.47778 & 70.91200 & 66.14583 & 70.61856 & 0.58498 & 501.70800 \\ \hline
  \textbf{Partición 2-1} & 95.08772 & 97.18310 & 0.50000 & 45.44200 & 66.11111 & 67.77778 & 0.48889 & 74.08400 & 62.37113 & 72.39583 & 0.46640 & 537.66100 \\ \hline
  \textbf{Partición 2-2} & 96.47887 & 98.94737 & 0.46667 & 45.10000 & 62.77778 & 75.55556 & 0.53333 & 77.48200 & 61.97917 & 65.97938 & 0.49012 & 496.93400 \\ \hline
  \textbf{Partición 3-1} & 96.14035 & 97.18310 & 0.53333 & 43.84000 & 73.33333 & 68.88889 & 0.48889 & 77.51000 & 63.91753 & 67.18750 & 0.49012 & 520.82900 \\ \hline
  \textbf{Partición 3-2} & 95.07042 & 98.24561 & 0.46667 & 45.26400 & 66.66667 & 78.88889 & 0.48889 & 77.85100 & 61.97917 & 67.52577 & 0.46640 & 492.01200 \\ \hline
  \textbf{Partición 4-1} & 95.08772 & 99.29577 & 0.40000 & 44.32100 & 73.33333 & 69.44444 & 0.47778 & 72.34400 & 68.04124 & 68.22917 & 0.54545 & 533.04200 \\ \hline
  \textbf{Partición 4-2} & 96.83099 & 97.19298 & 0.56667 & 44.96400 & 66.11111 & 73.88889 & 0.52222 & 72.25600 & 61.45833 & 69.58763 & 0.50593 & 503.29800 \\ \hline
  \textbf{Partición 5-1} & 96.84211 & 97.53521 & 0.33333 & 45.77500 & 65.55556 & 73.88889 & 0.53333 & 65.11200 & 60.30928 & 68.22917 & 0.52964 & 551.07800 \\ \hline
  \textbf{Partición 5-2} & 92.60563 & 98.24561 & 0.63333 & 49.35300 & 69.44444 & 65.55556 & 0.46667 & 65.12700 & 64.06250 & 65.97938 & 0.55731 & 518.51600 \\ \hline
  \textbf{Media} & 95.53546 & 98.13689 & 0.50000 & 45.03090 & 67.27778 & 71.55556 & 0.50111 & 72.23810 & 63.93363 & 68.39616 & 0.51107 & 519.59430 \\ \hline
  \textbf{Desv.Típica} & 1.20104 & 0.77188 & 0.09068 & 1.71898 & 3.44668 & 3.79083 & 0.02507 & 4.46933 & 2.78446 & 1.90630 & 0.03921 & 19.52783 \\ \hline
  \end{tabular}
  \end{adjustbox}
  \label{AGG}
  \end{table}
  
  \begin{table}[H]	
  \caption{Resultados del AGE}
  \begin{adjustbox}{width=1.1\textwidth}
  \begin{tabular}{|c|r|r|r|r|r|r|r|r|r|r|r|r|}
  \hline
  \multicolumn{1}{|l|}{} & \multicolumn{ 4}{c|}{\textbf{\textit{Wdbc}}} & \multicolumn{ 4}{c|}{\textbf{\textit{Movement\_Libras}}} & \multicolumn{ 4}{c|}{\textbf{\textit{Arrhytmia}}} \\ \hline
  \multicolumn{1}{|l|}{} & \multicolumn{1}{c|}{\textbf{\textit{\% clas test}}} & \multicolumn{1}{c|}{\textbf{\textit{\% clas train}}} & \multicolumn{1}{c|}{\textbf{\textit{tasa red}}} & \multicolumn{1}{c|}{\textbf{T(s)}} & \multicolumn{1}{c|}{\textbf{\textit{\% clas test}}} & \multicolumn{1}{c|}{\textbf{\textit{\% clas train}}} & \multicolumn{1}{c|}{\textbf{\textit{tasa red}}} & \multicolumn{1}{c|}{\textbf{T(s)}} & \multicolumn{1}{c|}{\textbf{\textit{\% clas test}}} & \multicolumn{1}{c|}{\textbf{\textit{\% clas train}}} & \multicolumn{1}{c|}{\textbf{\textit{tasa red}}} & \multicolumn{1}{c|}{\textbf{T(s)}} \\ \hline
  \textbf{Partición 1-1} & 95.43860 & 99.29577 & 0.40000 & 50.12100 & 65.00000 & 71.66667 & 0.57778 & 71.52400 & 67.52577 & 69.79167 & 0.49802 & 520.74700 \\ \hline
  \textbf{Partición 1-2} & 96.47887 & 98.24561 & 0.43333 & 52.61000 & 68.88889 & 69.44444 & 0.43333 & 78.16000 & 66.14583 & 68.55670 & 0.58498 & 474.08400 \\ \hline
  \textbf{Partición 2-1} & 95.43860 & 97.53521 & 0.60000 & 52.66800 & 72.22222 & 66.66667 & 0.47778 & 75.68800 & 64.43299 & 70.31250 & 0.51383 & 523.33000 \\ \hline
  \textbf{Partición 2-2} & 95.42254 & 98.59649 & 0.50000 & 47.40900 & 65.00000 & 72.77778 & 0.50000 & 75.94300 & 66.66667 & 69.58763 & 0.46640 & 503.43600 \\ \hline
  \textbf{Partición 3-1} & 95.78947 & 97.53521 & 0.60000 & 45.66200 & 73.88889 & 67.77778 & 0.50000 & 80.08000 & 61.85567 & 67.70833 & 0.52174 & 545.61800 \\ \hline
  \textbf{Partición 3-2} & 94.01408 & 98.24561 & 0.53333 & 47.37400 & 65.00000 & 79.44444 & 0.52222 & 75.07400 & 63.54167 & 66.49485 & 0.57708 & 482.20600 \\ \hline
  \textbf{Partición 4-1} & 94.73684 & 99.29577 & 0.40000 & 46.46900 & 72.22222 & 67.22222 & 0.42222 & 74.29900 & 63.91753 & 67.70833 & 0.47826 & 516.64900 \\ \hline
  \textbf{Partición 4-2} & 96.47887 & 97.54386 & 0.50000 & 47.50900 & 67.22222 & 72.77778 & 0.46667 & 78.42300 & 63.02083 & 68.55670 & 0.55336 & 490.41700 \\ \hline
  \textbf{Partición 5-1} & 96.49123 & 97.88732 & 0.50000 & 47.69300 & 66.66667 & 73.33333 & 0.54444 & 72.92700 & 62.88660 & 68.22917 & 0.50988 & 531.90500 \\ \hline
  \textbf{Partición 5-2} & 95.07042 & 98.24561 & 0.53333 & 45.56600 & 68.33333 & 68.33333 & 0.50000 & 70.38900 & 64.06250 & 71.13402 & 0.50988 & 493.30900 \\ \hline
  \textbf{Media} & 95.53595 & 98.24265 & 0.50000 & 48.30810 & 68.44444 & 70.94444 & 0.49444 & 75.25070 & 64.40561 & 68.80799 & 0.52134 & 508.17010 \\ \hline
  \textbf{Desv.Típica} & 0.77190 & 0.62919 & 0.06831 & 2.47796 & 3.14073 & 3.69392 & 0.04507 & 2.93907 & 1.72356 & 1.32211 & 0.03731 & 21.95454 \\ \hline  \end{tabular}
  \end{adjustbox}
  \label{AGE}
  \end{table}
  
  
  
  \begin{table}[H]
  \caption{Resultados globales}
  \begin{adjustbox}{width=1.1\textwidth}
  \begin{tabular}{|c|r|r|r|r|r|r|r|r|r|r|r|r|}
  \hline
  \multicolumn{1}{|l|}{} & \multicolumn{ 4}{c|}{\textbf{\textit{Wdbc}}} & \multicolumn{ 4}{c|}{\textbf{\textit{Movement\_Libras}}} & \multicolumn{ 4}{c|}{\textbf{\textit{Arrhytmia}}} \\ \hline
  \multicolumn{1}{|l|}{} & \multicolumn{1}{c|}{\textbf{\textit{\% clas test}}} & \multicolumn{1}{c|}{\textbf{\textit{\% clas train}}} & \multicolumn{1}{c|}{\textbf{\textit{tasa red}}} & \multicolumn{1}{c|}{\textbf{T(s)}} & \multicolumn{1}{c|}{\textbf{\textit{\% clas test}}} & \multicolumn{1}{c|}{\textbf{\textit{\% clas train}}} & \multicolumn{1}{c|}{\textbf{\textit{tasa red}}} & \multicolumn{1}{c|}{\textbf{T(s)}} & \multicolumn{1}{c|}{\textbf{\textit{\% clas test}}} & \multicolumn{1}{c|}{\textbf{\textit{\% clas train}}} & \multicolumn{1}{c|}{\textbf{\textit{tasa red}}} & \multicolumn{1}{c|}{\textbf{T(s)}} \\ \hline
  \textbf{3-NN} & 96.55597 & 96.55597 & 0.00000 & 0.00000 & 67.27778 & 68.22222 & 0.00000 & 0.00000 & 63.15883 & 63.31454 & 0.00000 & 0.00000 \\ \hline
  \textbf{SFS} & 93.63652 & 96.69644 & 0.87000 & 0.16150 & 64.50000 & 70.77778 & 0.89556 & 1.01970 & 69.85019 & 75.85911 & 0.98340 & 2.06640 \\ \hline
  \textbf{AGG} & 95.53546 & 98.13689 & 0.50000 & 45.03090 & 67.27778 & 71.55556 & 0.50111 & 72.23810 & 63.93363 & 68.39616 & 0.51107 & 519.59430 \\ \hline
  \textbf{AGE} & 95.53595 & 98.24265 & 0.50000 & 48.30810 & 68.44444 & 70.94444 & 0.49444 & 75.25070 & 64.40561 & 68.80799 & 0.52134 & 508.17010 \\ \hline
  \end{tabular}
  \end{adjustbox}
  \label{all}
  \end{table}
  
  Observamos a juzgar por los datos que estamos obteniendo que los algoritmos genéticos que hemos usado tienen la limitación
  de que su tasa de reducción tiende a ser 0.5 (ya que las soluciones iniciales se generan aleatoriamente, y tienden a tener
  la misma cantidad de ceros y de unos, por ser equiprobables los hechos de escoger un 0 o un 1. Además las operaciones que se
  hacen (mutaciones, cruces) conservan estas probabilidades de al escoger una característica ser 0 o 1. Por tanto, podemos
  encontrar ahí una limitación fuerte en los algoritmos genéticos por ejemplo en el dataset Arrhythmia, donde hemos observado
  que las mejores soluciones tienden a tener pocas características.
  
  \subsection{Wdbc}
  
  Es un problema de clasificación binaria en un dataset de 30 características y 569 instancias.
  
  \imagen{../data/genetic/wdbc.png}{Tasas de clasificación en Wdbc}{wdbcgraph}{0.7}
  
  
  Prácticamente los dos modelos de algoritmo genético funcionan igual de bien en todos los datasets. El \textit{overfiting} no
  se aprecia con tasas tan elevadas de \textit{score} en Wdbc. Observamos eso sí, que la desviación típica del modelo estacionario es
  menor tanto en porcentajes de clasificación como en tasas de acierto. Esto se debe a que la diversidad del modelo generacional
  es mayor, al ser sustituida la población entera por otra población en la que en el peor de los casos, todos los individuos
  podrían ser muy malas soluciones excepto una, que es la que se conserva por elitismo. En el modelo elitista, al ser únicamente
  dos los cromosomas que se van sustituyendo a cada paso en la población de 30, tenemos menor diversidad, pero una convergencia
  un poco más uniforme. Este comentario es extensible al análisis de los otros dos datasets.
  
  Las metaheurísticas que se han programado para este dataset (trayectorias, multiarranque, genéticos) no están consiguiendo
  batir al clasificador 3-NN. Este clasificador en baja dimensión aproxima muy bien el \textit{Bayes decision boundary}, que es
  el límite que estima el clasificador de Bayes para clasificar una instancia como de una clase u otra reduciendo la probabilidad
  de fallo. Podemos empezar a teorizar llegados a este punto, que en baja dimensión no vamos a conseguir obtener tasas muy
  superiores al clasificador 3-NN. En tema de interpretabilidad, las soluciones aportadas por los genéticos son mejores 
  que las proporcionadas por el clasificador 3-NN, pero peores que el SFS, ya que como se ha explicado la tasa de 
  reducción tiende a ser 0.5.
  
  En cuanto a tiempos, ambos genéticos tardan lo mismo, alrededor de 45 segundos. Constituyen tiempos bastante peores de 
  lo que tardaban las metaheurísticas de trayectorias o multiarranque, obteniendo tasas de reducción muy inferiores a las
  que se obtenían por ejemplo con el GRASP.
  %\imgn{../data/wdbc3nn.png}{Selección de características del SFS}{1}
  
  %\imgn{../data/wdbctabu.png}{Boxplot de variables del wdbc}{1}
 
  \subsection{Movement libras}
  
  Se trata de un problema de clasficación multiclase (15 clases después de normalizar el dataset). 
  Contiene 360 instancias y 90 características.
  
  Como se ha comentado, el generacional proporciona soluciones con mayor desviación típica en la tasa de clasificación y 
  de reducción.
  
  Para este dataset, solo se aprecia diferencia en tiempo con respecto a las metaheurísticas de multiarranque de la práctica
  anterior y en tasa de reducción respecto al GRASP, siendo peor en el caso de los genéticos.
  
  Para Movement Libras, el modelo estacionario funciona ligeramente mejor en cuanto a convergencia y tasa de clasificación
  de las soluciones que el generacional. Así bien las diferencias en términos porcentuales en \textit{score} entre ambos 
  modelos no son muy acusadas, apenas de un punto porcentual.
  
  \imagen{../data/genetic/mlibras.png}{Tasas de clasificación en Movement Libras}{wdbcgraph}{0.7}
  
  
  \subsection{Arrhythmia}
  
  Se trata de un problema de clasficación multiclase (5 clases después de normalizar el dataset). 
  Tiene 386 instancias y 253 características.
  
  
  Como se ha comentado, el generacional proporciona soluciones con mayor desviación típica en la tasa de clasificación y 
  de reducción.
  
  Prácticamente las dos metaheaurísticas son iguales, pero ofrecen tiempos muy superiores a los que arrojaban las 
  metaheaurísticas multiarranque, y unas tasas de reducción y de clasificación muy alejadas de la solución que obteníamos
  en el GRASP.
  
  Como punto positivo, podemos afirmar que los genéticos baten al clasificador vecino más cercano en este caso, pero no
  podemos olvidar que este último es especialmente malo en alta dimensionalidad. De hecho, el algoritmo que mejor funciona
  es el \textit{greedy} SFS, por seleccionar un número pequeño de características.
  
  \imagen{../data/genetic/arrhythmia.png}{Tasas de clasificación en Arrhythmia}{wdbcgraph}{0.7}
  
  \subsection{Notas finales}
  
  Podemos concluir que el uso de algoritmos genéticos en datasets de baja dimensionalidad está desaconsejado, por el tiempo
  que consume, que comparado al resto de metaheurísticas es grande.
  
  Al igual que en la práctica de multiarranque, en la que se propuso añadir a futuras prácticas un \textit{stepwise} para
  aumentar las tasas de reducción de las soluciones, durante la programación de los genéticos ha surgido la idea (que 
  plausiblemente pueda aplicarse en meméticos) de usar codificaciones con mayor probabilidad de contener ceros que unos y
  que esta probabilidad de generar un 0 o un 1 sea adaptativa al \textit{dataset} (características e instancias que tengamos).
  
  También se deja como intención el probar otros operadores de cruce con los algoritmos meméticos.
\end{document}